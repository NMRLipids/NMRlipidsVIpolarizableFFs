% \iffalse meta-comment
%<*internal>
\def\nameofplainTeX{plain}
\ifx\fmtname\nameofplainTeX\else
  \expandafter\begingroup
\fi
%</internal>
%<*install>
\input docstrip.tex
\keepsilent
\askforoverwritefalse
\preamble
----------------------------------------------------------------
achemso --- Support for submissions to American Chemical Society
  journals
E-mail: joseph.wright@morningstar2.co.uk
Released under the LaTeX Project Public License v1.3c or later
See http://www.latex-project.org/lppl.txt
----------------------------------------------------------------

\endpreamble
\postamble

Originally developed by Mats Dahlgren
Copyright (C) 1996-1998 by Mats Dahlgren
Copyright (C) 2008-2019 by
  Joseph Wright <joseph.wright@morningstar2.co.uk>
 
Part of this bundle is derived from cite.sty, to which the
following license applies:
  Copyright (C) 1989-2009 by Donald Arseneau
  These macros may be freely transmitted, reproduced, or
  modified provided that this notice is left intact.

It may be distributed and/or modified under the conditions of
the LaTeX Project Public License (LPPL), either version 1.3c of
this license or (at your option) any later version.  The latest
version of this license is in the file:

   http://www.latex-project.org/lppl.txt

This work is "maintained" (as per LPPL maintenance status) by
  Joseph Wright.

This work consists of the files achemso.dtx,
                                achemso-demo.bib and
                                achemso-demo.tex,
          and the derived files achemso.cls,
                                achemso.ins,
                                achemso.pdf,
                                achemso.sty,
                                natmove.sty and
                                a number of configuration files.
                               
\endpostamble
\usedir{tex/latex/achemso}
\generate{
  \file{\jobname.cls}{\from{\jobname.dtx}{class}}
  \file{\jobname.sty}{\from{\jobname.dtx}{package}}
  \file{natmove.sty}{\from{\jobname.dtx}{natmove}}
}
\usedir{tex/latex/achemso/config}
\generate{\file{achemso-aaembp.cfg}{\from{\jobname.dtx}{aaembp}}
          \file{achemso-aaemcq.cfg}{\from{\jobname.dtx}{aaemcq}}
          \file{achemso-aamick.cfg}{\from{\jobname.dtx}{aamick}}
          \file{achemso-aanmf6.cfg}{\from{\jobname.dtx}{aanmf6}}
          \file{achemso-aapmcd.cfg}{\from{\jobname.dtx}{aapmcd}}
          \file{achemso-abseba.cfg}{\from{\jobname.dtx}{abseba}}
          \file{achemso-accacs.cfg}{\from{\jobname.dtx}{accacs}}
          \file{achemso-achre4.cfg}{\from{\jobname.dtx}{achre4}}
}
\generate{\file{achemso-acscii.cfg}{\from{\jobname.dtx}{acscii}}
          \file{achemso-acbcct.cfg}{\from{\jobname.dtx}{acbcct}}
          \file{achemso-acncdm.cfg}{\from{\jobname.dtx}{acncdm}}
          \file{achemso-acsccc.cfg}{\from{\jobname.dtx}{acsccc}}
          \file{achemso-aesccq.cfg}{\from{\jobname.dtx}{aesccq}}
          \file{achemso-aelccp.cfg}{\from{\jobname.dtx}{aelccp}}
          \file{achemso-ascefj.cfg}{\from{\jobname.dtx}{ascefj}}
          \file{achemso-amlccd.cfg}{\from{\jobname.dtx}{amlccd}}
}
\generate{\file{achemso-amlcef.cfg}{\from{\jobname.dtx}{amlcef}}
          \file{achemso-amclct.cfg}{\from{\jobname.dtx}{amclct}}
          \file{achemso-ancac3.cfg}{\from{\jobname.dtx}{ancac3}}
          \file{achemso-acsodf.cfg}{\from{\jobname.dtx}{acsodf}}
          \file{achemso-apchd5.cfg}{\from{\jobname.dtx}{apchd5}}
          \file{achemso-aidcbc.cfg}{\from{\jobname.dtx}{aidcbc}}
          \file{achemso-ascecg.cfg}{\from{\jobname.dtx}{ascecg}}
          \file{achemso-asbcd6.cfg}{\from{\jobname.dtx}{asbcd6}}
}
\generate{\file{achemso-ancham.cfg}{\from{\jobname.dtx}{ancham}}
          \file{achemso-bichaw.cfg}{\from{\jobname.dtx}{bichaw}}
          \file{achemso-bcches.cfg}{\from{\jobname.dtx}{bcches}}
          \file{achemso-bomaf6.cfg}{\from{\jobname.dtx}{bomaf6}}
          \file{achemso-crtoec.cfg}{\from{\jobname.dtx}{crtoec}}
          \file{achemso-chreay.cfg}{\from{\jobname.dtx}{chreay}}
          \file{achemso-cmatex.cfg}{\from{\jobname.dtx}{cmatex}}
          \file{achemso-cgdefu.cfg}{\from{\jobname.dtx}{cgdefu}}
}
\generate{\file{achemso-enfuem.cfg}{\from{\jobname.dtx}{enfuem}}
          \file{achemso-esthag.cfg}{\from{\jobname.dtx}{esthag}}
          \file{achemso-estlcu.cfg}{\from{\jobname.dtx}{estlcu}}
          \file{achemso-iecred.cfg}{\from{\jobname.dtx}{iecred}}
          \file{achemso-inoraj.cfg}{\from{\jobname.dtx}{inoraj}}
          \file{achemso-jafcau.cfg}{\from{\jobname.dtx}{jafcau}}
          \file{achemso-jceda8.cfg}{\from{\jobname.dtx}{jceda8}}
          \file{achemso-jceaax.cfg}{\from{\jobname.dtx}{jceaax}}
}
\generate{\file{achemso-jcisd8.cfg}{\from{\jobname.dtx}{jcisd8}}
          \file{achemso-jctcce.cfg}{\from{\jobname.dtx}{jctcce}}
          \file{achemso-jmcmar.cfg}{\from{\jobname.dtx}{jmcmar}}
          \file{achemso-jnprdf.cfg}{\from{\jobname.dtx}{jnprdf}}
          \file{achemso-joceah.cfg}{\from{\jobname.dtx}{joceah}}
          \file{achemso-jpcafh.cfg}{\from{\jobname.dtx}{jpcafh}}
          \file{achemso-jpcbfk.cfg}{\from{\jobname.dtx}{jpcbfk}}
          \file{achemso-jpccck.cfg}{\from{\jobname.dtx}{jpccck}}
}
\generate{\file{achemso-jpclcd.cfg}{\from{\jobname.dtx}{jpclcd}}
          \file{achemso-jprobs.cfg}{\from{\jobname.dtx}{jprobs}}
          \file{achemso-jacsat.cfg}{\from{\jobname.dtx}{jacsat}}
          \file{achemso-langd5.cfg}{\from{\jobname.dtx}{langd5}}
          \file{achemso-mamobx.cfg}{\from{\jobname.dtx}{mamobx}}
          \file{achemso-mpohbp.cfg}{\from{\jobname.dtx}{mpohbp}}
          \file{achemso-nalefd.cfg}{\from{\jobname.dtx}{nalefd}}
          \file{achemso-orlef7.cfg}{\from{\jobname.dtx}{orlef7}}
}
\generate{\file{achemso-oprdfk.cfg}{\from{\jobname.dtx}{oprdfk}}
          \file{achemso-orgnd7.cfg}{\from{\jobname.dtx}{orgnd7}}
}
\usedir{bibtex/bst/achemso}
\generate{
  \file{achemso.bst}{\from{\jobname.dtx}{bst}}
  \file{biochem.bst}{\from{\jobname.dtx}{bst,bio}}
}
%</install>
%<install>\endbatchfile
%<*internal>
\usedir{source/latex/achemso}
\generate{
  \file{\jobname.ins}{\from{\jobname.dtx}{install}}
}
\ifx\fmtname\nameofplainTeX
  \expandafter\endbatchfile
\else
  \expandafter\endgroup
\fi
%</internal>
%<*package|class|natmove>
\NeedsTeXFormat{LaTeX2e}
%</package|class|natmove>
%<*driver>
\documentclass[a4paper]{ltxdoc}
\usepackage{achemso,array,booktabs,lmodern}
\usepackage[osf]{mathpazo}
\usepackage[scaled=0.95]{helvet}
\usepackage[final]{listings,microtype}
\usepackage[T1]{fontenc}
\usepackage[numbered]{hypdoc}
\EnableCrossrefs
\CodelineIndex
\RecordChanges
\begin{document}
  \DocInput{\jobname.dtx}
\end{document}
%</driver>
% \fi
%
%\GetFileInfo{\jobname.sty}
%
%\makeatletter
%
%^^A \DescribeOption is in l3doc but not ltxdoc
%\newcommand*\DescribeOption{^^A
%  \leavevmode
%  \@bsphack
%  \begingroup
%    \MakePrivateLetters
%    \Describe@Option
%}
%\newcommand*\Describe@Option[1]{^^A
%    \endgroup
%  \marginpar{^^A
%    \raggedleft
%    \PrintDescribeEnv{#1}^^A
%  }%
%  \SpecialOptionIndex{#1}^^A
%  \@esphack
%  \ignorespaces
%}
%\newcommand*\SpecialOptionIndex[1]{^^A
%  \@bsphack
%  \index{^^A
%    #1\actualchar{\protect\ttfamily#1} (option)\encapchar usage^^A
%  }^^A
%  \index{^^A
%    options:\levelchar#1\actualchar{\protect\ttfamily#1}
%    \encapchar usage^^A
%  }^^A
%  \@esphack
%}
%
%^^A For creating examples with nice highlighting of code, and so
%^^A on; based on the system used in the listings source (lstsample).
%\lst@RequireAspects{writefile}
%\newsavebox{\LaTeXdemo@box}
%\lstnewenvironment{LaTeXdemo}[1][code and example]{^^A
%  \global\let\lst@intname\@empty
%  \expandafter\let\expandafter\LaTeXdemo@end
%    \csname LaTeXdemo@#1@end\endcsname
%  \@nameuse{LaTeXdemo@#1}^^A
%}{^^A
%  \LaTeXdemo@end
%}
%\newcommand*\LaTeXdemo@new[3]{^^A
%  \expandafter\newcommand\expandafter*\expandafter
%    {\csname LaTeXdemo@#1\endcsname}{#2}^^A
%  \expandafter\newcommand\expandafter*\expandafter
%    {\csname LaTeXdemo@#1@end\endcsname}{#3}^^A
%}
%\newcommand*\LaTeXdemo@common{^^A
%  \setkeys{lst}{
%    basicstyle   = \small\ttfamily,
%    basewidth    = 0.51em,
%    gobble       = 3,
%    keywordstyle = \color{blue},
%    language     = [LaTeX]{TeX},
%    moretexcs    = {
%      affiliation,
%      alsoaffiliation,
%      altaffiliation,
%      email,
%      fax,
%      natmovechars,
%      phone,
%      setkeys
%    }
%  }^^A
%}
%\newcommand*\LaTeXdemo@input{^^A
%  \MakePercentComment
%  \catcode`\^^M=10\relax
%  \small
%  \begingroup
%    \setkeys{lst}{
%      SelectCharTable=\lst@ReplaceInput{\^\^I}{\lst@ProcessTabulator}
%    }^^A
%    \leavevmode
%      \input{\jobname.tmp}^^A
%  \endgroup
%  \MakePercentIgnore
%}
%\LaTeXdemo@new{code and example}{^^A
%  \setbox\LaTeXdemo@box=\hbox\bgroup
%    \lst@BeginAlsoWriteFile{\jobname.tmp}^^A
%    \LaTeXdemo@common
%}{^^A
%    \lst@EndWriteFile
%  \egroup
%  \begin{center}
%    \ifdim\wd\LaTeXdemo@box>0.48\linewidth\relax
%      \hbox to\linewidth{\box\LaTeXdemo@box\hss}^^A
%        \begin{minipage}{\linewidth}
%          \LaTeXdemo@input
%        \end{minipage}
%    \else
%      \begin{minipage}{0.48\linewidth}
%        \LaTeXdemo@input
%      \end{minipage}
%      \hfill
%      \begin{minipage}{0.48\linewidth}
%        \hbox to\linewidth{\box\LaTeXdemo@box\hss}^^A
%      \end{minipage}
%    \fi
%  \end{center}
%}
%\LaTeXdemo@new{code only}{^^A
%  \LaTeXdemo@common
%}{^^A
%}
%
%\providecommand*\eTeX{\ensuremath{\varepsilon}-\TeX}
%\providecommand*\file{\texttt}
%\providecommand*\opt{\texttt}
%\providecommand*\pkg{\textsf}
%
%\newcommand*\ACS{\textsc{acs}}
%
%\makeatother
%
%\title{^^A
%  \textsf{achemso} --- Support for submissions to American Chemical
%    Society journals\thanks{^^A
%    This file describes version \fileversion, last revised
%    \filedate.^^A
%  }^^A
%}
%\author{^^A
%  Joseph Wright\thanks{E-mail: joseph.wright@morningstar2.co.uk}^^A
%}
%\date{Released \filedate}
%
%\maketitle
%
% \changes{v1.0}{1998/06/01}{Initial release of package by Mats Dahlgren}
% \changes{v2.0}{2007/01/17}{Re-write of package by Joseph Wright}
% \changes{v3.0}{2008/07/20}{^^A
%   Second re-write, converting to a class and giving much tighter
%   integration with \textsc{acs} submission system^^A
% }
% \changes{v3.1}{2009/04/21}{^^A
%   Re-ordered code to reduce or remove dependency on other packages^^A
% }
% \changes{v3.1}{2008/11/08}{Added \cs{alsoaffiliation} macro}
% \changes{v3.2}{2009/04/20}{^^A
%   New functions \cs{SectionsOff}, \cs{SectionsOn},
%   \cs{SectionNumbersOff} and \cs{SectionNumbersOn}
% }
% \changes{v3.2}{2009/04/21}{New \texttt{tocentry} section}
% \changes{v3.2}{2009/04/20}{^^A
%   New functions \cs{AbstractOff} and \cs{AbstractOn}^^A
% }
% \changes{v3.2c}{2009/05/29}{Added \cs{fax} macro}
% \changes{v3.2c}{2009/05/29}{Added \cs{phone} macro}
% \changes{v3.2b}{2009/05/05}{New \texttt{email} option}
% \changes{v3.2c}{2009/05/28}{Fix so that \texttt{email} option is turned
%   on by default}
% \changes{v3.3}{2008/08/28}{\cs{altaffiliation} macro checks for
%   duplicate entries and uses only one footnote}
% \changes{v3.3}{2009/08/28}{New \texttt{layout} option}
% \changes{v3.3a}{2009/09/01}{New \texttt{etalmode} option}
% \changes{v3.4}{2010/01/15}{More detail on installation}
% \changes{v3.5}{2010/06/25}{New \BibTeX{} style file}
% \changes{v3.5a}{2010/08/05}{Modify JACS Communication to print
%   abstract}
% \changes{v3.5c}{2010/10/11}{Update \emph{Org.~Lett.}\ style}
% \changes{v3.5c}{2010/10/11}{Bug fix for \cs{mciteSubRef} macro}
% \changes{v3.4}{2010/01/15}{Require \pkg{cleveref}, dropping
%   \pkg{varioref} use}
% \changes{v3.6}{2012/01/11}{Drop \pkg{cleverref} use}
% \changes{v3.7}{2012/04/30}{Rename \opt{usetitle} option to
%   \opt{articletitle}}
% \changes{v3.7}{2012/04/30}{New \opt{chaptertitle} option for bibliography}
% \changes{v3.8}{2013/08/23}{Extend journal list to current ACS range}
% \changes{v3.8}{2013/08/16}{Introduce \cs{latin} command to support
%   \emph{ACS Nano} requirements}
% \changes{v3.8a}{2013/09/04}{Minor style improvements for some
%  journals}
% \changes{v3.8b}{2013/09/15}{Improve output ordering for \emph{J.~Phys.\
%   Chem.\ Lett.}}
% \changes{v3.8c}{2013/09/20}{Improve \emph{ACS Photonics} style}
% \changes{v3.8d}{2013/10/04}{Fix bug in keyword printing}
% \changes{v3.8e}{2014/01/08}{Minor style changes for \emph{J.~Med.~Chem.}
%  and \emph{ACS Appl.~Mater.}}
% \changes{v3.8g}{2014/03/06}{Fix issue with \emph{J.~Phys.\ Chem.}\ formatting}
% \changes{v3.8i}{2014/05/14}{Fix issue with \emph{Chem.\ Res.\ Toxicol.}\ formatting}
% \changes{v3.8j}{2014/06/12}{Update article title requirements for all journals}
% \changes{v3.8j}{2014/06/12}{Update keyword requirements for all journals}
% \changes{v3.8j}{2014/06/12}{Remove \emph{Biotechnol.\ Prog.}\ configuration}
% \changes{v3.8k}{2014/08/18}{Format date correctly for in press bibliography
%   entries}
% \changes{v3.8l}{2014/08/23}{Fix printing of phone/fax numbers}
% \changes{v3.9}{2015/02/01}{New \texttt{doi} option}
% \changes{v3.9a}{2015/03/03}{Update style for \emph{ACS Central Sci.}}
% \changes{v3.9b}{2015/03/22}{Update \emph{J. Agric. Food Chem.} style}
% \changes{v3.10}{2015/04/07}{Update address printing to reflect new
%   ACS style}
% \changes{v3.10}{2015/04/07}{Add starred (subs)sections to TOC}
% \changes{v3.10a}{2015/04/21}{Add article titles for \emph{Chem.\ Mater.}
%   and \emph{Chem. Rev.}}
% \changes{v3.10c}{2016/05/10}{Add support for \emph{ACS Omega} and
%   \emph{ACS Energy Lett.}}
% \changes{v3.10d}{2016/06/17}{Improve handling of \textsc{doi} values}
% \changes{v3.10e}{2016/08/31}{Tweak \emph{Macromolecules} formatting}
% \changes{v3.10f}{2016/09/07}{Various formatting updates}
% \changes{v3.10h}{2017/01/14}{Add style for \emph{ACS Biomater.\ Sci.\ Eng.}}
% \changes{v3.10h}{2017/01/20}{Add style for \emph{ACS Earth Space Chem.}}
% \changes{v3.10h}{2017/01/20}{Add style for \emph{ACS Infect.\ Dis.}}
% \changes{v3.10h}{2017/01/20}{Add style for \emph{ACS Sensors}}
% \changes{v3.10h}{2017/01/20}{Style update for \emph{J.~Chem.\ Theory.\
%   Comput.}}
% \changes{v3.10h}{2017/01/20}{Update supplementary information text}
% \changes{v3.10i}{2017/02/09}{Style update for \emph{Biochemistry}}
% \changes{v3.10i}{2017/05/18}{Fix issue with \cs{doi} command}
% \changes{v3.10i}{2017/05/18}{Updates for \emph{Org.\ Lett.}}
% \changes{v3.11}{2018/01/07}{Print article titles for all journals}
% \changes{v3.11b}{2018/07/12}{Correct handling of \texttt{and others} in
%   bibliography}
% \changes{v3.11c}{2018/07/12}{Further work on bibliography}
%
%\begin{abstract}
% The \pkg{achemso} bundle provides a \LaTeX\ class file and \BibTeX\
% style file in accordance with the requirements of the American
% Chemical Society (\ACS).  The files can be used for any documents, but
% have been carefully designed and tested to be suitable for submission
% to \ACS\ journals.
%
% The bundle also includes the \pkg{natmove} package.  This package is
% loaded by \pkg{achemso}, and provides automatic moving of superscript
% citations after punctuation.
%\end{abstract}
%
%\begin{multicols}{2}
%  \tableofcontents
%\end{multicols}
%
%\section{Introduction}
% Support for \BibTeX\ bibliography following the requirements of the
% American Chemical Society (\ACS), along with a package to make
% these easy to  have been available since version one of \pkg{achemso}.
% The re-write from version 1 to version 2 made a number of
% improvements to the package, and also added a number of new
% features.  However, neither version one nor version two of the
% package was targeted directly at use for submissions to \ACS\
% journals.  This new release of \pkg{achemso} addresses this issue.
%
% The bundle consists of four parts.  The first is a \LaTeXe\ class,
% intended for use in submissions.  It is based on the standard
% \pkg{article} class, but makes various changes to facilitate ease
% of use.  The second part is the \LaTeX\ package.  The package contains
% the parts of the bundle which are appropriate for use with other
% document classes.\footnote{For example, when writing a thesis.} 
% Thirdly, two \BibTeX\ style files are included.  These are used by
% both the class and the package, but can be used directly if desired.
% Finally, an example document is included; this is intended to act a
% potential template for submission, and illustrates the use of the
% class file.
%
%\section{Installation}
%
% The package is supplied in \file{dtx} format and as a pre-extracted
% zip file, \file{\jobname.tds.zip}. The later is most convenient for
% most users: simply unzip this in your local texmf directory and
% run \texttt{texhash} to update the database of file locations. If
% you want to unpack the \file{dtx} yourself, running
% \texttt{tex \jobname.dtx} will extract the package whereas
% \texttt{latex \jobname.dtx} will extract it and also typeset the
% documentation.
%
% Typesetting the documentation requires a number of packages in
% addition to those needed to use the package. This is mainly
% because of the number of demonstration items included in the text. To
% compile the documentation without error, you will need the packages:
% \begin{itemize}
%   \item \pkg{array}
%   \item \pkg{booktabs}
%   \item \pkg{hypdoc}
%   \item \pkg{listings}
%   \item \pkg{lmodern}
%   \item \pkg{mathpazo}
%   \item \pkg{microtype}
%\end{itemize}
%
%\section{Requirements}
%
% The \pkg{achemso} class requires the following packages:
%\begin{itemize}
%  \item \pkg{caption}
%  \item \pkg{float}
%  \item \pkg{geometry}
%  \item \pkg{natbib}
%  \item \pkg{setspace}
%  \item \pkg{xkeyval}
%\end{itemize}
% These are normally present in the current major \TeX{} distributions,
% but are also available from \href{http://www.ctan.org}{The
% Comprehensive TeX Archive Network}.
%
%\section{The class file}
% The class file has been designed for use in submitting journals to
% the \ACS. It uses all of the modifications described here (those in
% the package as well as those in the class).  The accompanying
% example manuscript can be used as a template for the correct use of
% the class file.  It is intended to act as a model for submission.
%
% When submitting communications to \emph{J.~Am.\ Chem.\ Soc.}, the
% class will automatically lay the document out in the publication
% style. This allows the author to judge the length of text submitted
% more accurately.  Changing the \opt{manuscript} in the
% demonstration document to \opt{communication} will illustrate the
% effect.
%
%\subsection{Class options}
%
%\DescribeOption{journal}
% The class supports a limited number of options, which are
% specifically-targeted at submission.  The class uses the
% \pkg{keyval} system for options, in the form \opt{key=value}. The
% most important option is \opt{journal}.  This is the name of the
% target journal for the publication.  The package is designed such
% that the choice of journal will set up the correct bibliography
% style and so on.  The journals currently recognised by the package
% are summarised in Table~\ref{tab:journal}.  If an unknown journal
% is specified, the package will fall-back on the
% \opt{journal=jacsat} option.
%\begin{table}
%  \centering
%  \begin{tabular}{@{}>{\itshape}l>{\ttfamily}l>{\itshape}l>{\ttfamily}l@{}}
%    \toprule
%      Journal & \rmfamily Setting & Journal & \rmfamily Setting \\
%    \midrule
%      Acc.\ Chem.\ Res.               & achre4 &
%      Chem.\ Res.\ Toxicol.           & crtoec \\
%      ACS Appl.\ Energy Mater.        & aaemcq &
%      Chem.\ Rev.                     & chreay \\
%      ACS Appl.\ Electron.\ Mater.    & aaembp &
%      Cryst.\ Growth Des.             & cgdefu \\
%      ACS Appl.\ Mater.\ Interfaces   & aamick &
%      Energy Fuels                    & enfuem \\
%      ACS Appl.\ Nano Mater.          & aanmf6 &
%      Environ.\ Sci.\ Technol.        & esthag \\
%      ACS Appl.\ Polym.\ Mater.       & aapmcd &
%      Environ.\ Sci.\ Technol.\ Lett. & estlcu \\
%      ACS Biomater.\ Sci.\ Eng.       & abseba &
%      Ind.\ Eng.\ Chem.\ Res.         & iecred \\
%      ACS Catal.                      & accacs &
%      Inorg.\ Chem.                   & inoraj \\
%      ACS Central Sci.                & acscii &
%      J.~Agric.\ Food Chem.           & jafcau \\
%      ACS Chem.\ Biol.                & acbcct &
%      J.~Chem.\ Ed.                   & jceda8 \\
%      ACS Chem.\ Neurosci.            & acncdm &
%      J.~Chem.\ Eng.\ Data            & jceaax \\
%      ACS Combinatorial Sci.          & acsccc &
%      J.~Chem.\ Inf.\ Model.          & jcisd8 \\
%      ACS Earth Space Chem.           & aesccq &
%      J.~Chem.\ Theory Comput.        & jctcce \\
%      ACS Energy Lett.                & aelccp &
%      J.~Med.\ Chem.                  & jmcmar \\
%      ACS Infect.\ Dis.               & ascefj &
%      J.~Nat.\ Prod.                  & jnprdf \\
%      ACS Macro Lett.                 & amlccd &
%      J.~Org.\ Chem.                  & joceah \\
%      ACS Mater.\ Lett.               & amlcef \\
%      J.~Phys.\ Chem.~A               & jpcafh \\
%      ACS Med.\ Chem.\ Lett.          & amclct &
%      J.~Phys.\ Chem.~B               & jpcbfk \\
%      ACS Nano                        & ancac3 &
%      J.~Phys.\ Chem.~C               & jpccck \\
%      ACS Omega                       & acsodf &
%      J.~Phys.\ Chem.\ Lett.          & jpclcd \\
%      ACS Photon.                     & apchd5 &
%      J.~Proteome Res.                & jprobs \\
%      ACS Sensors                     & ascefj &
%      J.~Am.\ Chem.\ Soc.             & jacsat \\
%      ACS Sustainable Chem.\ Eng.     & ascecg &
%      Langmuir                        & langd5 \\
%      ACS Synth.\ Biol.               & asbcd6 &
%      Macromolecules                  & mamobx \\
%      Anal.\ Chem.                    & ancham &
%      Mol.\ Pharm.                    & mpohbp \\
%      Biochemistry                    & bichaw &
%      Nano Lett.                      & nalefd \\
%      Bioconjugate Chem.              & bcches &
%      Org.\ Lett.                     & orlef7 \\
%      Biomacromolecules               & bomaf6 &
%      Organometallics                 & orgnd7 \\
%      Chem.\ Mater.                   & cmatex \\
%    \bottomrule
%  \end{tabular}
%  \caption{Values for \opt{journal} option}
%  \label{tab:journal}
%\end{table}
%
%\DescribeOption{manuscript}
% The second option is the \opt{manuscript} option. This specifies
% the type of paper in the manuscript.  The values here are
% \opt{article}, \opt{note}, \opt{communication}, \opt{review},
% \opt{letter} and \opt{perspective}. The valid values will depend on
% the value of \opt{journal}.  The \opt{manuscript} option determines
% whether sections and an abstract are valid.  The value
% \opt{suppinfo} is also available for supporting information.
%
%\DescribeOption{layout}
% The \pkg{achemso} class can produce drafts in two layout styles. The
% standard setting for the \opt{layout} option is \opt{traditional},
% which produces a double-spaced single column manuscript. The
% alternative setting \opt{twocolumn} will use single spacing and
% print the text in two columns. The second option is obviously more
% compact. If the journal requires a particular style this option may
% be ignored.
%
%\DescribeOption{email}
% It may be desirable to omit e-mail addresses from the front page of
% a manuscript. The printing of e-mail addresses can be disabled using
% the \opt{email} option, which takes Boolean values only. The default
% is to print e-mail addresses. Notice that phone and fax numbers are
% only printed if e-mail addresses are printed.
%
% Other options are provided by the package, but when used with the
% class these are silently ignored. If you need to override the
% settings chosen by the class, include the settings \emph{after}
% the \cs{documentclass} line using \cs{setkeys}:
%\begin{LaTeXdemo}[code only]
%  \documentclass[journal = jacsat]{achemso}
%  \setkeys{acs}{articletitle = false}
%\end{LaTeXdemo}
%
%\subsection{Manuscript meta-data}
%
%\DescribeMacro{\author}
%\DescribeMacro{\affiliation}
%\DescribeMacro{\alsoaffiliation}
%\DescribeMacro{\altaffiliation}
%\DescribeMacro{\email}
% Inspired by REV\TeX, the \pkg{achemso} class alters the method for
% adding author information to the manuscript.  Each author should be
% given as a separate \cs{author} command.  These should be followed
% by an \cs{affiliation}, which applies to the preceding authors. The
% \cs{affiliation} macro takes an optional argument, for a short
% version of the affiliation.\footnote{This will usually be the
% university or company name.}  At least one author should be
% followed by an \cs{email} macro, containing contact details.  All
% authors with an e-mail address are automatically marked with a
% star.  The example manuscript demonstrates the use of all of these
% macros. Notice that \cs{alsoaffiliation} is used when one (or more)
% authors work at multiple institutions, while \cs{altaffiliation}
% is intended for previous addresses (or other notes). Only
% \cs{affiliation} applies to multiple authors: both
% \cs{alsoaffiliation} and \cs{altaffiliation} are set on a per
% author basis.
%\begin{LaTeXdemo}[code only]
%  \author{Author Person}
%  \author{Second Bloke}
%  \email{second.bloke@some.place}
%  \affiliation[University of Sometown]
%    {University of Somewhere, Sometown, USA}
%  \altaffiliation
%    {Previous address: Minute University, Nowhereville, USA}
%  \author{Indus Trialguy}
%  \email{i.trialguy@sponsor.co}
%  \affiliation[SponsoCo]
%    {Research Department, SponsorCo, BigCity, USA}
%  \alsoaffiliation{University of Somewhere, Sometown, USA}
%\end{LaTeXdemo}
% Repeated entries for \cs{affiliation} and \cs{alsoaffiliation} will
% result in only one address being printed in the address list and
% footnotes. Repeated \cs{altaffiliation} entries only produce a single
% footnote, and can therefore be used for entries such as
%\begin{LaTeXdemo}[code only]
%  \author{First Coworker}
%  \altaffiliation{Contributed equally to this work}
%  \author{Second Coworker}
%  \altaffiliation{Contributed equally to this work}
%\end{LaTeXdemo}
% if required.
%
%\DescribeMacro{\fax}
%\DescribeMacro{\phone}
% The class will recognise the optional information \cs{fax} and
% \cs{phone}, which will be printed along with the lead authors
% e-mail address. Note that this information is only used for authors
% who have an e-mail address supplied.
%\begin{LaTeXdemo}[code only]
%  \author{Second Bloke}
%  \email{second.bloke@some.place}
%  \phone{+xxx (0)yyy zzzzzz}
%  \fax{+xxx (0)yyy wwwwww}
%  \affiliation[University of Sometown]
%    {University of Somewhere, Sometown, USA}
%\end{LaTeXdemo}
%
%\DescribeMacro{\and}
%\DescribeMacro{\thanks}
% The method used for setting the meta-data means that the normal
% \cs{and} and \cs{thanks} macros are not appropriate in the \
% pkg{achemso} class.  Both produce a warning if used.
%
%\DescribeMacro{\title}
% The \cs{title} macro is extended to accept an optional argument.
% This is intended for a shortened version of the journal title
% suitable for running headers. Some journals require that authors
% supply this data: if it is needed then it will be printed in the
% appropriate place.
%\begin{LaTeXdemo}[code only]
%  \title[Short running title]
%    {Long title which would not fit in running headers}
%\end{LaTeXdemo}
%
% The meta-data items should be given in the preamble to the \LaTeX\
% file, and no \cs{maketitle} macro is required in the document body.
% This is all handled by the class file directly.  At least one
% author, affiliation and e-mail address must be specified.
%
%\subsection{Floats}
%\DescribeEnv{scheme}
%\DescribeEnv{chart}
%\DescribeEnv{graph}
% The class defines three new floating environments: \texttt{scheme},
% \texttt{chart} and \texttt{graph}. These can be used as expected to
% include graphical content.  The placement of these new floats and the
% standard \texttt{table} and \texttt{figure} floats is altered to be
% ``here'' if possible.  The contents of all floats is automatically
% horizontally centred on the page.
%
%\subsection{Section headers}
%
%\DescribeMacro{\SectionNumbersOff}
%\DescribeMacro{\SectionNumbersOn}
% Some journals have no section numbering by default. This can be
% set up in the appropriate configuration file, but it may be that
% individual users need to change the decision. The macros
% \cs{SectionNumbersOff} and \cs{SectionNumbersOn} are therefore
% available: these should be given in the preamble.
%
%\DescribeMacro{\SectionsOff}
%\DescribeMacro{\SectionsOn}
% More radically, the entire availability of sections can be turned on
% and of. This is functionality is available to the user \emph{via} the
% \cs{SectionsOn} and \cs{SectionsOff} macros, which again are
% preamble-only.
%\DescribeMacro{\AbstractOff}
%\DescribeMacro{\AbstractOn}
% Similar functions are available for the abstract: \cs{AbstractOff} and
% \cs{AbstractOn}.
%
%\subsection{Special sections}
%
%\DescribeEnv{acknowledgement}
%\DescribeEnv{suppinfo}
% The sections for acknowledgements and supporting information
% have dedicated environments available.  These ensure that
% the section headings are generated, and that the text is
% sized corrected when using creating a Communication.
%\begin{LaTeXdemo}[code only]
%  \begin{acknowledgement}
%    The authors thank A.N.~Other.
%  \end{acknowledgement}
%
%  \begin{suppinfo}
%    Full characterization data for all new compounds.
%  \end{suppinfo}
%\end{LaTeXdemo}
%
%\DescribeEnv{tocentry}
% For generating an entry for the graphical table of content, required
% by some journals, the environment \texttt{tocentry} is available. This
% prints its content in an appropriately sized box on a separate page.
% In contrast to the rest of the manuscript, this section is intended to
% be ``print ready'' in appearance.
%
%\begin{LaTeXdemo}[code only]
%  \begin{tocentry}
%    \includegraphics{toc-entry-graphic}
%    Some text to explain the graphic.
%  \end{tocentry}
%\end{LaTeXdemo}
%
% \subsection{Miscellaneous commands}
% 
% \DescribeMacro{\latin}
%   
% The command \cs{latin} is provided by the class to format Latin phrases
% such as ``et al.'' Most ACS journals do not make these italic, but
% some (for example \emph{ACS Nano}) do. By providing a command, the formatting
% is left flexible.
%
% \DescribeMacro{\doi}
%
% The bibliography style prints any \textsc{doi} values as the
% argument to the command \cs{doi}. The default definition will allow
% printing of special characters but does not create hyperlinks. A
% more sophisticated version of the command may be set up if \pkg{hyperref}
% is loaded, for example
%\begin{LaTeXdemo}[code only]
%  \begin{tocentry}
%    \newcommand{\doi}[1]{\href{http://dx.doi.org/#1}{\nolinkurl{#1}}}
%  \end{tocentry}
%\end{LaTeXdemo}
%
%\section{The package file}
% The \pkg{achemso} package is independent of the class file, and
% contains parts of the bundle useful outside of submission to the
% \ACS.
%
%\subsection{Package options}
%
% As with the class options, the package uses the key--value method
% for option set up. These are used to control the output of citations
% and bibliographic data. The same options are used when creating
% journal configurations for the class: this is a task most users will
% not need to undertake!
%
%\DescribeOption{super}
% The \opt{super} option affects the handling of superscript
% reference markers.  The option switches this behaviour
% on and off (and takes Boolean values: \opt{super=true} and
% \opt{super=false} are valid).
%
%\DescribeOption{articletitle}
% The \opt{articletitle} option is a Boolean, and sets whether the title
% of a paper referenced appears in the bibliography.  The default is
% \opt{articletitle=true}.
%
% \DescribeOption{doi}
% The boolean \opt{doi} option is provided to allow a \textsc{doi}
% (Digital Object Identifier) to be included for bibliography entries
% even where other identification such as page numbers is available.
%  The standard setting is \opt{false}: setting it to \opt{true} will
%  cause \textsc{doi} numbers to be printed if available.
%
% \DescribeOption{chaptertitle}
% The boolean \opt{chaptertitle} option is provided to allow flexibility
% for the inclusion of chapter titles for \texttt{book} and
% related entries. The standard setting is \opt{false}: setting
% it to \opt{true} will cause chapter titles to be included.
%
%\DescribeOption{etalmode}
%\DescribeOption{maxauthors}
% Many journals require that long lists of authors are shortened using
% `et al.' in the references section. The behaviour of the \BibTeX\
% styles in this regard is controlled by two options, \opt{etalmode}
% and \opt{maxauthors}. There are two possible ways to shorten a long
% list of authors. Some journals require that only the first author
% is given, followed by `et al.': for this behaviour, set
% \opt{etalmode=firstonly}:
%\begin{verbatim}
%  Jones, A.N. et al.
%\end{verbatim}
% On the other hand, some journals request that the list of authors
% is truncated after $n$ authors. This is set up by the
% \opt{etalmode=truncate} option:
%\begin{verbatim}
%  Jones, A.N.; Bloggs, F.; Nobacon, D. et al.
%\end{verbatim}
% In both cases, the maximum number of authors permitted before
% introducing `et al.' is governed by the \opt{maxauthors} option.
% This option recognises the sentinel value $0$, which indicates that
% no shortening should occur at all.
%
%\DescribeOption{biblabel}
% Redefining the formatting of the numbers used in the bibliography
% usually requires modifying internal \LaTeX\ macros.  The
% \opt{biblabel} option makes these changes more accessible: valid
% values are \opt{plain} (use the number only), \opt{brackets}
% (surround the number in brackets) and \opt{period} or
% \opt{fullstop} (follow the number by a full stop/period).
%
%\DescribeOption{biochem}
%\DescribeOption{biochemistry}
% Most \ACS\ journals use the same bibliography style, with the only
% variation being the inclusion of article titles.  However, a small
% number of journals use a rather different style; the journal
% \emph{Biochemistry} is probably the most prominent.  The
% \opt{biochemistry} or \opt{biochem} option uses the style of
% \emph{Biochemistry} for the bibliography, rather than the normal
% \ACS\ style.
%
%\subsection{Bibliography notes}
%
%\DescribeMacro{\bibnote}
%\DescribeMacro{\bibnotemark}
%\DescribeMacro{\bibnotetext}
% \pkg{achemso} provides the \cs{bibnote} macro. This is intended for
% addition of notes to the bibliography (references).  The macro
% accepts a single argument, which is transferred to the bibliography by
% \BibTeX. In analogy to \cs{footnote}, the macros \cs{bibnotemark} and
% \cs{bibnotetext} are available for dividing up the marker for a
% note from the text.
%\begin{LaTeXdemo}
%  Some text \bibnote{This note text will be in the bibliography}.
%\end{LaTeXdemo}
%
% The functionality for bibnotes in \pkg{achemso} is based on that
% in the \pkg{notes2bib} package. The \pkg{notes2bib} package can be
% loaded with the \pkg{achemso} package, and no clash will occur. With
% the class file, \pkg{notes2bib} will be ignored if requested, to
% prevent issues on submission to the \ACS.
%
%\section{The \texorpdfstring{\BibTeX}{BibTeX} style files}
%
% \pkg{achemso} is supplied with two style files, \file{achemso.bst} and
% \file{biochem.bst}.  The direct use of these without the \pkg{achemso}
% package file is not recommended, but is possible.  The style files
% can be loaded in the usual way, with a \cs{bibliographystyle}
% command.  The \pkg{natbib} package must be loaded by the \LaTeX\ file
% concerned, if the \pkg{achemso} package is not in use.
%
% The style files are designed to use the \pkg{mciteplus} package if it
% is available, but to work even if is not. When \pkg{mciteplus} is
% present, it is possible to automatically produce references of the
% form
%\begin{quotation}
%  \noindent
%  (5) (a) Arduengo,~A.~J.,~III; Dias,~H. V.~R.; Harlow,~R.~L.;
%  Kline,~M. \emph{{J}.~{A}m.\ {C}hem.\ {S}oc.} \textbf{1992},
%  \emph{114}, 5530--5534; (b) Appelhans,~L.~N.; Zuccaccia,~D.;
%  Kovacevic,~A.; Chianese,~A.~R.; Miecznikowski,~J.~R.; Macchioni,~A.;
%  Clot,~E.; Eisenstein,~O.; Crabtree,~R.~H. \emph{{J}.~{A}m.\ {C}hem.
%  {S}oc.} \textbf{2005}, \emph{127}, 16299--16311; (c)
%  Arduengo,~A.~J.,~III; Gamper,~S.~F.; Calabrese,~J.~C.; Davidson,~F.
%  \emph{J.~Am.\ Chem.\ Soc.} \textbf{1994}, \emph{116}, 4391--4394.
%\end{quotation}
% as demonstrated in the example document. When \pkg{mciteplus} is
% not present, this functionality is not available but the style files
% will work normally.
%
% The \BibTeX\ style files implement the bibliographic style
% specified by the \ACS\ in \emph{The ACS Style Guide}
% \cite{Coghill2006}.  By default, article titles are not included in
% output using the \file{achemso.bst} file, but are with the
% \file{biochem.bst} file.
%
% The style used by the \ACS\ does not differentiate between
% \BibTeX\ \texttt{book}, \texttt{inbook}, \texttt{collection} and
% \texttt{incollection} entries. As a result, the appearance of
% these entry types depends on the fields available. Named subdivisions
% of a book (for example, chapters where each has a named author)
% should be given in the \texttt{title} field, with the title of
% the book itself in the \texttt{booktitle} field. The
% \texttt{chapter} field should be used for a chapter number, and
% is printed as part of the pagination.
%
% One frequently asked question is why some people see an empty first
% item in their bibliography when using the \pkg{achemso} package or
% class. This is usually because they have downloaded the \LaTeX\ files
% and done a local installation without also updating the \BibTeX\
% style. The two must be from the same version of \pkg{achemso}: they
% are designed to work together.
%
%\section{The \pkg{natmove} package}
%
% The \pkg{natmove} package does only one job.  It brings the ability to
% move punctuation after citations, using code borrowed from the
% \pkg{cite} package. Both the \pkg{achemso} class and package load
% \pkg{natmove} automatically.
%\begin{LaTeXdemo}
%  Some text \cite{Coghill2006} some more text.\\
%  Some text ending a sentence \cite{Coghill2006}.
%\end{LaTeXdemo}
% This is deactivated for other citation types.
%\begin{LaTeXdemo}
%  Some text \citeyear{Coghill2006}.\\
%  Some text \citeauthor{Coghill2006}.\\
%  Some text \citenum{Coghill2006}.
%\end{LaTeXdemo}
% The package does nothing if the \opt{super} option has not been
% given to \pkg{natbib}.  This means that the source can be written
% without needing to decide where citations will to appear, with the
% \opt{super} option for \pkg{natbib} controlling the result.
%
%\DescribeMacro{\natmovechars}
% One user macro is provided: \cs{natmovechars}.  This contains
% the characters which are moved before superscript punctuation.
% The default contents is |,;:.| and can be set using
% \cs{renewcommand*}:
%\begin{LaTeXdemo}
%  \renewcommand*{\natmovechars}{.}
%  Some text \cite{Coghill2006},
%  more text \cite{Coghill2006}.
%\end{LaTeXdemo}
%
%\StopEventually{^^A
%  \PrintChanges
%  \PrintIndex
%  \bibliography{achemso-demo}
%}
%
%\section{Implementation}
%
% A lot of the work done by the package is also needed by the class.
% Loading the package and the class makes load-order awkward. Instead,
% the two parts are done in one place. Sandwiching the common code are
% two slices of dedicated material for the class and the package. Some
% of this is needed ``early'', before the common material, whilst the
% rest is ``late''.
%
%\subsection{Early class-only code}
%
% So that there is no confusion, the base class is loaded early.
%    \begin{macrocode}
%<*class>
\ProvidesClass{achemso}
  [2019-02-14 v3.12a Submission to ACS journals]
\LoadClass[12pt,letter]{article}
%    \end{macrocode}
%   
%\begin{macro}{\acs@warning}
% The code for a warning is created so that it works for the package too.
%    \begin{macrocode}
\newcommand*\acs@warning{\ClassWarning{achemso}}
%</class>
%    \end{macrocode}
%\end{macro}
%
%\subsection{Early package-only code}
%
% The package and the class should not both be loaded, as the two use
% the same internal macro names. On the other hand, if the class is
% not in use a reminder is printed to use it if possible.
%    \begin{macrocode}
%<*package>
\ProvidesPackage{achemso}
  [2019-02-14 v3.12a Support for ACS journals]
\@ifclassloaded{achemso}{%
  \PackageInfo{achemso}{%
    You have already loaded the `achemso' class:\MessageBreak
    loading the package will abort%
  }%
  \endinput
}{%
  \PackageInfo{achemso}{%
    When writing a submission to an ACS journal, please\MessageBreak
    use the achemso document class%
  }%
}
%    \end{macrocode}
%   
%\begin{macro}{\acs@warning}
% The code for a warning is created so that it works for the class too.
%    \begin{macrocode}
\newcommand*\acs@warning{\PackageWarning{achemso}}
%</package>
%    \end{macrocode}
%\end{macro}
%   
%\subsection{Common code}
%
%\begin{macro}{\acs@ifundefined}
% A non-expandable test for defined macros: does not add to the hash
% table.
%    \begin{macrocode}
%<*package|class>
\newcommand*\acs@ifundefined[1]{%
  \begingroup\expandafter\expandafter\expandafter\endgroup
  \expandafter\ifx\csname #1\endcsname\relax
    \expandafter\@firstoftwo
  \else
    \expandafter\@secondoftwo
  \fi
}
%    \end{macrocode}
%\end{macro}
%
% The first stage needed is to read the package options given. Although
% \pkg{xkeyval} was perhaps not the best choice, changing this now would
% be rather risky.
%    \begin{macrocode}
\RequirePackage{xkeyval}
%    \end{macrocode}
%\begin{macro}{\acs@keyval@bool}
% A support macro for making Boolean options: the \pkg{xkeyval} version
% is only available in newer releases.
%    \begin{macrocode}
\newcommand*\acs@keyval@bool[2]{%
  \acs@ifundefined{acs@#1#2}{%
    \acs@warning{Unknown option `#2' for key #1}%
  }{%
    \@nameuse{acs@#1#2}%
  }%
}
%    \end{macrocode}
%\end{macro}
%\begin{macro}{\ifacs@abbreviations}
%\begin{macro}{\ifacs@articletitle}
%\begin{macro}{\ifacs@biochem}
%\begin{macro}{\ifacs@chaptertitle}
%\begin{macro}{\ifacs@doi}
%\begin{macro}{\ifacs@email}
%\begin{macro}{\ifacs@hyperref}
%\begin{macro}{\ifacs@keywords}
%\begin{macro}{\ifacs@super}
% These are all very trivial definitions: to avoid issues with older
% versions of \pkg{xkeyval} each definition is done directly.
%    \begin{macrocode}
\newif\ifacs@abbreviations
\newif\ifacs@articletitle
\newif\ifacs@biochem
\newif\ifacs@doi
\newif\ifacs@chaptertitle
\newif\ifacs@email
\newif\ifacs@hyperref
\newif\ifacs@keywords
\newif\ifacs@super
\define@key{acs}{abbreviations}[true]{%
  \acs@keyval@bool{abbreviations}{#1}%
}
\define@key{acs}{articletitle}[true]{%
  \acs@keyval@bool{articletitle}{#1}%
}
\define@key{acs}{biochem}[true]{%
  \acs@keyval@bool{biochem}{#1}%
}
\define@key{acs}{doi}[true]{%
  \acs@keyval@bool{doi}{#1}%
}
\define@key{acs}{chaptertitle}[true]{%
  \acs@keyval@bool{chaptertitle}{#1}%
}
\define@key{acs}{email}[true]{%
  \acs@keyval@bool{email}{#1}%
}
\define@key{acs}{hyperref}[true]{%
  \acs@keyval@bool{hyperref}{#1}%
}
\define@key{acs}{keywords}[true]{%
  \acs@keyval@bool{keywords}{#1}%
}
\define@key{acs}{super}[true]{%
  \acs@keyval@bool{super}{#1}%
}
\define@key{acs}{usetitle}[true]{%
  \acs@keyval@bool{articletitle}{#1}%
}
%    \end{macrocode}
%\end{macro}
%\end{macro}
%\end{macro}
%\end{macro}
%\end{macro}
%\end{macro}
%\end{macro}
%\end{macro}
%\end{macro}
%\begin{macro}{\acs@journal}
%\begin{macro}{\acs@layout}
%\begin{macro}{\acs@manuscript}
%\begin{macro}{\acs@maxauthors}
% Trivial again: done without using \pkg{xkeyval} for the same reasons
% as before.
%    \begin{macrocode}
\newcommand*\acs@journal{jacsat}
\newcommand*\acs@layout{traditional}
\newcommand*\acs@manuscript{article}
\newcommand*\acs@maxauthors{15}
\define@key{acs}{journal}{%
  \def\acs@journal{#1}%
}
\define@key{acs}{layout}{%
  \def\acs@layout{#1}%
}
\define@key{acs}{manuscript}{%
  \def\acs@manuscript{#1}%
}
\define@key{acs}{maxauthors}{%
  \def\acs@maxauthors{#1}%
}
%    \end{macrocode}
%\end{macro}
%\end{macro}
%\end{macro}
%\end{macro}
%\begin{macro}{\ifacs@etal@truncate}
%\begin{macro}{\acs@etal@firstonly}
%\begin{macro}{\acs@etal@truncate}
% The setup for the \opt{etalmode} option is quite simple: just look
% for the appropriate macros.
%    \begin{macrocode}
\newif\ifacs@etal@truncate
\define@key{acs}{etalmode}{%
  \acs@ifundefined{acs@etal@#1}{%
    \acs@warning{%
      Unknown value `#1' for\MessageBreak  etalmode option%
    }%
  }{%
    \@nameuse{acs@etal@#1}%
  }%
}
\newcommand*\acs@etal@firstonly{\acs@etal@truncatefalse}
\newcommand*\acs@etal@truncate{\acs@etal@truncatetrue}
%    \end{macrocode}
%\end{macro}
%\end{macro}
%\end{macro}
% \begin{macro}{\acs@activate@biblabel}
% The \opt{biochemistry} option is an alias for \opt{biochem}. The
% \opt{biblabel} option is a choice, which is implemented using a
% csname search. The group here prevents hash table pollution, whilst
% the \pkg{xkeyval} method is avoided as it is more complex than it is
% worth!
%    \begin{macrocode}
\define@key{acs}{biochemistry}{%
  \setkeys{acs}{biochem = #1}%
}
\define@key{acs}{biblabel}{%
  \acs@ifundefined{acs@biblabel@#1}{%
    \acs@warning{%
      Unknown value `#1' for\MessageBreak biblabel option%
    }%
  }{%
    \acs@activate@biblabel{\@nameuse{acs@biblabel@#1}}%
  }%
}
\newcommand*\acs@activate@biblabel{}
%<*class>
\let\acs@activate@biblabel\AtEndOfClass
\AtEndOfClass{\let\acs@activate@biblabel\@firstofone}
%</class>
%<*package>
\let\acs@activate@biblabel\AtEndOfPackage
\AtEndOfPackage{\let\acs@activate@biblabel\@firstofone}
%</package>
%    \end{macrocode}
% \end{macro}
%\begin{macro}{\acs@biblabel@brackets}
%\begin{macro}{\acs@biblabel@fullstop}
%\begin{macro}{\acs@biblabel@period}
%\begin{macro}{\acs@biblabel@plain}
% The macros to implement the idea above for biblabels.
%    \begin{macrocode}
\newcommand*\acs@biblabel@brackets{\def\bibnumfmt##1{(##1)}}
\newcommand*\acs@biblabel@fullstop{\def\bibnumfmt##1{##1.}}
\newcommand*\acs@biblabel@period{\def\bibnumfmt##1{##1.}}
\newcommand*\acs@biblabel@plain{\def\bibnumfmt##1{##1}}
%    \end{macrocode}
%\end{macro}
%\end{macro}
%\end{macro}
%\end{macro}
% Set up some defaults.
%    \begin{macrocode}
\setkeys{acs}{
  email = true,
  super = true
}
%    \end{macrocode}
% Loading some other packages depends on the options chosen, so they
% are processed now.
%    \begin{macrocode}
\ProcessOptionsX*<acs>
%    \end{macrocode}
%   
%\begin{macro}{\acs@manuscript@communication}
%\begin{macro}{\acs@manuscript@letter}
%\begin{macro}{\acs@manuscript@note}
%\begin{macro}{\acs@manuscript@review}
%\begin{macro}{\acs@manuscript@suppinfo}
% For text comparisons.
%    \begin{macrocode}
\newcommand*\acs@manuscript@communication{communication}
\newcommand*\acs@manuscript@letter{letter}
\newcommand*\acs@manuscript@note{note}
\newcommand*\acs@manuscript@review{review}
\newcommand*\acs@manuscript@suppinfo{suppinfo}
%    \end{macrocode}
%\end{macro}
%\end{macro}
%\end{macro}
%\end{macro}
%\end{macro}
%   
%\begin{macro}{\acs@niib@create}
%\begin{macro}{bibnote}
%\begin{macro}{\thebibnote}
%\begin{macro}{\bibnote}
%\begin{macro}{\bibnotemark}
%\begin{macro}{\bibnotetext}
%\begin{macro}{\printbibnotes}
% To avoid needing to load the \pkg{notes2bib} package, especially as
% the plan is to move that package to \LaTeX3 internal syntax,
% \pkg{achemso} provides a minimal version here. The first step is to
% create macros which will need a guard against \pkg{notes2bib} already
% having been loaded. To allow the package and class to behave
% differently these are actually applied later.
%    \begin{macrocode}
\newcommand*\acs@niib@create{%
  \@namedef{ver@notes2bib.sty}{%
    2009/04/20 v1.6a Integrating notes into the bibliography (achemso
    version)
  }%
  \@ifundefined{c@bibnote}{\newcounter{bibnote}}{}
  \def\thebibnote{%
    Note-\the\value{bibnote}%
  }%
  \DeclareRobustCommand*{\bibnote}[1][\thebibnote]{%
    \stepcounter{bibnote}%
    \def\acs@niib@after@text{\cite{##1}}%
    \acs@niib@text{##1}%
  }%
  \DeclareRobustCommand*{\bibnotemark}[1][\thebibnote]{%
    \stepcounter{bibnote}%
    \cite{##1}%
  }%
  \DeclareRobustCommand*{\bibnotetext}[1][\thebibnote]{%
    \let\acs@niib@after@text\relax
    \acs@niib@text{##1}%
  }%
  \newcommand*\printbibnotes{%
    \ifnum\the\value{bibnote}>\z@\relax
      \nocite{achemso-control}%
      \acs@bibliography{acs-\jobname}%
    \fi
  }%
}
%    \end{macrocode}
%\end{macro}
%\end{macro}
%\end{macro}
%\end{macro}
%\end{macro}
%\end{macro}
%\end{macro}
%\begin{macro}{\acs@niib@after@text}
% After the text.
%    \begin{macrocode}
\newcommand*\acs@niib@after@text{}
%    \end{macrocode}
%\end{macro}
%\begin{macro}{\acs@niib@text}
% The \cs{acs@niib@text} macro is the outer part of the writing system.
% It does not absorb the text of note, as without \eTeX\ this is bad
% news. The same file is used for notes and the control entry for
% the bibliography style.
%    \begin{macrocode}
\newcommand*\acs@niib@text{%
  \@bsphack
  \if@filesw
    \expandafter\acs@niib@write
  \else
    \expandafter\acs@niib@no@write
  \fi
}
%    \end{macrocode}
%\end{macro}
%\begin{macro}{\acs@niib@write}
%\begin{macro}{\acs@niib@write@aux@i}
%\begin{macro}{\acs@niib@write@aux@ii}
% Writing verbatim without \eTeX.
%    \begin{macrocode}
\newcommand*\acs@niib@write[1]{%
  \begingroup
    \let\do\@makeother
    \dospecials
    \catcode`\{ 1\relax
    \catcode`\} 2\relax
    \acs@niib@write@aux@i{#1}%
}
\newcommand*\acs@niib@write@aux@i[1]{%
  \long\def\@tempa##1{%
      \def\@tempa{##1}%
      \@onelevel@sanitize\@tempa
    \expandafter\endgroup
    \expandafter\def\expandafter\@tempa\expandafter{\@tempa}%
    \acs@niib@write@aux@ii{#1}%
  }%
  \catcode`\^^M 10\relax
  \@tempa
}
\newcommand*\acs@niib@write@aux@ii[1]{%
  \immediate\write\acs@bib@file{%
    @Misc\string{#1,^^J%
    \space\space note = \string{\@tempa\string},^^J%
    \string}^^J%
  }%
  \@esphack
  \acs@niib@after@text
}
%    \end{macrocode}
%\end{macro}
%\end{macro}
%\end{macro}
%\begin{macro}{\acs@niib@no@write}
% If no files are to be written, a bit of tidying up.
%    \begin{macrocode}
\newcommand\acs@niib@no@write[2]{%
  \@esphack
  \acs@niib@after@text
}
%    \end{macrocode}
%\end{macro}
%
%\begin{macro}{\nmv@natbib@detect}
%\begin{macro}{\acs@nmv@activate}
%\begin{macro}{\acs@autonote}
% The functionality of \pkg{notes2bib} is combined with the standard
% \cs{cite} macro, to give automatic note-like data in the bibliography.
%    \begin{macrocode}
\AtBeginDocument{
  \def\nmv@natbib@detect{%
    \ifNAT@super
      \expandafter\acs@nmv@activate
    \else
      \expandafter\acs@autonote
    \fi
  }
}
\newcommand*\acs@nmv@activate{%
  \let\nmv@citex@nat\@citex
  \let\@citex\nmv@citex
  \let\nmv@cite\cite
  \renewcommand*{\cite}[2][]{%
    \nmv@ifmtarg{##1}{%
      \nmv@citetrue
      \nmv@cite{##2}%
    }{%
      \nocite{##2}%
      \bibnote{Ref.~\citenum{##2}, ##1}%
    }%
  }%
}
\newcommand*\acs@autonote{%
  \let\nmv@cite\cite
  \renewcommand*{\cite}[2][]{%
    \nmv@ifmtarg{##1}{%
      \nmv@cite{##2}%
    }{%
      \nocite{##2}%
      \bibnote{Ref.~\citenum{##2}, ##1}%
    }%
  }%
}
%    \end{macrocode}
%\end{macro}
%\end{macro}
%\end{macro}
%
%\begin{macro}{\acs@bib@file}
%\begin{macro}{\acs@bib@message}
%\begin{macro}{\acs@bib@name}
% Some information or creating the control file for \BibTeX\ is set up.
%    \begin{macrocode}
\newwrite\acs@bib@file
\newcommand*\acs@bib@message{%
  This is an auxiliary file used by the `achemso' bundle.^^J%
  This file may safely be deleted. It will be recreated as required.^^J
}
\newcommand*\acs@bib@name{acs-\jobname.bib}
%    \end{macrocode}
%\end{macro}
%\end{macro}
%\end{macro}
%\begin{macro}{\acs@bib@write}
%\begin{macro}{\acs@bib@write@aux}
% The control information for \BibTeX\ needs to be written to a special
% file. The main writing macro is quite simple. Actually writing the
% information is left to the code for \cs{bibliography}, so that this
% only happens if needed.
%    \begin{macrocode}
\newcommand*\acs@bib@write{%
  \if@filesw
    \expandafter\acs@bib@write@aux
  \fi
}
\AtBeginDocument{\acs@bib@write}
\newcommand*\acs@bib@write@aux{%
  \immediate\openout\acs@bib@file\acs@bib@name\relax
  \immediate\write\acs@bib@file{\acs@bib@message}%
  \edef\@tempa##1##2{%
    \space\space##1\space = "##2",^^J%
  }%
  \immediate\write\acs@bib@file{%
    @Control\string{%
      achemso-control,^^J%
      \@tempa{ctrl-article-title\space}{%
        \ifacs@articletitle yes\else no\fi
      }%
      \@tempa{ctrl-chapter-title\space}{%
        \ifacs@chaptertitle yes\else no\fi
      }%
      \@tempa{ctrl-doi\space\space\space\space\space\space\space
        \space\space\space\space}{%
        \ifacs@doi yes\else no\fi
      }%
      \@tempa{ctrl-etal-number\space\space\space}{\acs@maxauthors}%
      \@tempa{ctrl-etal-firstonly}{%
        \ifacs@etal@truncate no\else yes\fi
      }%
    \string}^^J%
  }%
  \immediate\write\@auxout{%
    \string\citation\string{achemso-control\string}%
  }%
  \AtEndDocument{%
    \immediate\closeout\acs@bib@file\relax
  }%
}
%    \end{macrocode}
%\end{macro}
%\end{macro}
%
%\begin{macro}{\acs@bibliography}
%\begin{macro}{\bibliography}
% The \cs{bibliography} macro is now patched so that everything works
% correctly.
%    \begin{macrocode}
\AtBeginDocument{%
  \let\acs@bibliography\bibliography
  \def\bibliography#1{%
    \acs@bibliography{acs-\jobname,#1}%
  }%
}
%    \end{macrocode}
%\end{macro}
%\end{macro}
%
% \begin{macro}{\latin}
%   The journal \emph{ACS Nano} formats Latin phrases differently from
%   every other ACS journal: we provide a \cs{latin} command to cover this.
%    \begin{macrocode}
\AtBeginDocument{
  \providecommand{\latin}[1]{#1}
}
%</package|class>
%    \end{macrocode}
% \end{macro}
%
%\subsection{Late class-only code}
%
% Most of the power of the class is now created. First, a few options
% are reset so that any given by the user are effectively ignored.
%    \begin{macrocode}
%<*class>
\setkeys{acs}{
  abbreviations = false,
  articletitle  = true,
  biblabel      = brackets,
  biochem       = false,
  doi           = false,
  etalmode      = firstonly,
  keywords      = false,
  maxauthors    = 15,
  super         = true
}
%    \end{macrocode}
% When using the class, \pkg{notes2bib} is always emulated. Other
% standard support packages can now be loaded.
%    \begin{macrocode}
\acs@niib@create
\RequirePackage[margin=2.54cm]{geometry}
\RequirePackage{
  caption,
  float,
  graphicx,
  setspace,
  url
}
\ifacs@hyperref
  \expandafter\RequirePackage
\else
  \expandafter\@gobble
\fi
  {hyperref}
\AtBeginDocument{\doublespacing}
%    \end{macrocode}
%   
%\begin{macro}{\title}
%\begin{macro}{\@title}
%\begin{macro}{\acs@title@short}
% For the meta-data, the REV\TeX\ bundle provides a good model for
% the commands to give the author.  \cs{gdef} is used here to
% avoid any odd grouping issues. 
%    \begin{macrocode}
\renewcommand*{\title}[2][]{%
  \gdef\acs@title@short{#1}%
  \gdef\@title{#2}%
  \ifx\acs@title@short\@empty
    \global\let\acs@title@short\@title
  \fi
}
\@onlypreamble\title
%    \end{macrocode}
%\end{macro}
%\end{macro}
%\end{macro}
%\begin{macro}{\acs@author@cnt}
%\begin{macro}{\acs@affil@cnt}
%\begin{macro}{\acs@affil@alt@cnt}
% \pkg{achemso} tracks the number authors, affiliations and alternative
% affiliations.
%    \begin{macrocode}
\newcount\acs@author@cnt
\newcount\acs@affil@cnt
\newcount\acs@affil@alt@cnt
%    \end{macrocode}
%\end{macro}
%\end{macro}
%\end{macro}
%\begin{macro}{\acs@footnote@cnt}
%\begin{macro}{\acs@affil@marker@cnt}
% Two counts for getting affiliation footnotes correct.
%    \begin{macrocode}
\newcount\acs@footnote@cnt
\newcount\acs@affil@marker@cnt
%    \end{macrocode}
%\end{macro}
%\end{macro}
%\begin{macro}{\author}
% The author macro stores the current author details and sets the
% affiliation of the author to the current one. Everything is
% \cs{global} so that there is no possibility of begin trapped inside a
% group. The affiliation counter is always one behind, and so it is
% locally incremented to keep the logic of the code clear elsewhere.
%    \begin{macrocode}
\def\author#1{%
  \global\advance\acs@author@cnt\@ne\relax
  \expandafter\gdef\csname @author@\@roman\acs@author@cnt\endcsname{#1}%
  \begingroup
    \advance\acs@affil@cnt\@ne
    \expandafter\xdef\csname @author@affil@\@roman
      \acs@author@cnt\endcsname
      {\the\acs@affil@cnt}%
  \endgroup
}
\@onlypreamble\author
%    \end{macrocode}
%\end{macro}
%\begin{macro}{\and}
%\begin{macro}{\thanks}
% Neither \cs{and} nor \cs{thanks} are used by the document class.
%    \begin{macrocode}
\def\and{%
  \acs@warning{%
    \string\and\space not used by the achemso class: please see
    the\MessageBreak package documentation for details%
  }%
}
\def\thanks{%
  \acs@warning{%
    \string\thanks\space not used by the achemso class: please see
    the\MessageBreak the package documentation for details%
  }%
}
%    \end{macrocode}
%\end{macro}
%\end{macro}
%\begin{macro}{\affiliation}
% As with \cs{author}, everything is \cs{global} just in case. The
% system insists that affiliations come after authors. Before anything
% is committed, a check is made that the affiliation has not already
% been seen.
%    \begin{macrocode}
\newcommand*\affiliation[2][\relax]{%
  \ifnum\acs@author@cnt>\z@\relax
    \acs@affil@ifdup{#2}{%
      \acs@affil@swap{#2}%
    }{%
      \global\advance\acs@affil@cnt\@ne\relax
      \expandafter\gdef\csname @address@\@roman\acs@affil@cnt\endcsname
        {#2}%
      \ifx\relax#1\relax
        \expandafter\gdef\csname @affil@\@roman\acs@affil@cnt\endcsname
          {#2}%
      \else
        \expandafter\gdef\csname @affil@\@roman\acs@affil@cnt\endcsname
          {#1}%
      \fi
    }%
  \else
    \acs@warning{Affiliation with no author}%
  \fi
}
\@onlypreamble\affiliation
%    \end{macrocode}
%\end{macro}
%\begin{macro}{\acs@affil@ifdup}
%\begin{macro}{\acs@affil@ifdup@aux}
% A short test for two addresses being identical.
%    \begin{macrocode}
\newcommand*\acs@affil@ifdup[1]{%
  \begingroup
    \def\@tempa{#1}%
    \@tempswafalse
    \@tempcnta\z@\relax
    \acs@affil@ifdup@aux
    \if@tempswa
      \aftergroup\@firstoftwo
    \else
      \aftergroup\@secondoftwo
    \fi
  \endgroup
}
\newcommand*\acs@affil@ifdup@aux{%
  \advance\@tempcnta\@ne\relax
  \expandafter\expandafter\expandafter\def\expandafter\expandafter
    \expandafter\@tempb\expandafter\expandafter\expandafter
    {\csname @address@\@roman\@tempcnta\endcsname}%
  \ifx\@tempa\@tempb
    \expandafter\@tempswatrue
  \else
    \ifnum\@tempcnta<\acs@affil@cnt\relax
      \expandafter\expandafter\expandafter\acs@affil@ifdup@aux
    \fi
  \fi
}
%    \end{macrocode}
%\end{macro}
%\end{macro}
%\begin{macro}{\acs@affil@swap}
%\begin{macro}{\acs@affil@swap@aux}
% If the affiliation has already been given, then all of the authors
% need to be checked to make sure that the correct affiliation is used.
% First, the loop from above is used to find the correct number for the
% duplicate.
%    \begin{macrocode}
\newcommand*\acs@affil@swap[1]{%
  \begingroup
    \def\@tempa{#1}%
    \@tempcnta\z@\relax
    \@tempcntb\z@\relax
    \acs@affil@ifdup@aux
    \advance\acs@affil@cnt\@ne\relax
    \acs@affil@swap@aux
  \endgroup
}
\newcommand*\acs@affil@swap@aux{%
  \advance\@tempcntb\@ne\relax
  \expandafter\ifnum\csname @author@affil@\@roman\@tempcntb\endcsname
    = \acs@affil@cnt\relax
    \expandafter\xdef\csname @author@affil@\@roman\@tempcntb\endcsname{%
      \the\@tempcnta
    }%
  \fi
  \ifnum\@tempcntb<\acs@author@cnt\relax
    \expandafter\acs@affil@swap@aux
  \fi
}
%    \end{macrocode}
%\end{macro}
%\end{macro}
%\begin{macro}{\alsoaffiliation}
%\begin{macro}{\acs@alsoaffil@find}
% To allow complex affiliations , two commands are needed. The first
% deals with affiliations that are in some way shared by several
% authors. This is tracked on a per author basis.
%    \begin{macrocode}
\newcommand*\alsoaffiliation[2][\relax]{%
  \ifnum\acs@author@cnt>\z@\relax
    \acs@affil@ifdup{#2}{%
      \acs@alsoaffil@find{#2}%
    }{%
      \global\advance\acs@affil@cnt\@ne\relax
      \@tempcnta\acs@affil@cnt\relax
      \expandafter\gdef\csname @address@\@roman\acs@affil@cnt\endcsname
        {#2}%
      \ifx\relax#1\relax
        \expandafter\gdef\csname @affil@\@roman\acs@affil@cnt\endcsname
          {#2}%
      \else
        \expandafter\gdef\csname @affil@\@roman\acs@affil@cnt\endcsname
          {#1}%
      \fi
    }%
    \@ifundefined{@author@alsoaffil@\@roman\acs@author@cnt}{%
      \expandafter\xdef\csname @author@alsoaffil@\@roman\acs@author@cnt
        \endcsname{\the\@tempcnta}%
    }{%
      \expandafter\xdef\csname @author@alsoaffil@\@roman\acs@author@cnt
        \endcsname{%
        \csname @author@alsoaffil@\@roman\acs@author@cnt\endcsname
        ,\the\@tempcnta
      }%
    }%
  \else
    \acs@warning{Affiliation with no author}%
  \fi
}
\newcommand*\acs@alsoaffil@find[1]{%
  \begingroup
    \def\@tempa{#1}%
    \@tempcnta\z@\relax
    \@tempcntb\z@\relax
    \acs@affil@ifdup@aux
  \expandafter\endgroup
  \expandafter\@tempcnta\the\@tempcnta\relax
}
%    \end{macrocode}
%\end{macro}
%\end{macro}
%\begin{macro}{\altaffiliation}
% For the alternative affiliation, a second count is kept, and the
% affiliation is ``attached'' to the author. The way these are stored
% means that the appropriate affiliation number can be recovered later,
% and so printed correctly when things get complex.
%    \begin{macrocode}
\newcommand*\altaffiliation[1]{%
  \ifnum\acs@author@cnt>\z@\relax
    \begingroup
      \acs@altaffil@ifdup{#1}{%
        \expandafter\xdef\csname @author@altaffil@\@roman\acs@author@cnt
          \endcsname{\the\@tempcnta}%
      }{%
        \global\advance\acs@affil@alt@cnt\@ne\relax
        \expandafter\gdef\csname @altaffil@\@roman\acs@affil@alt@cnt
          \endcsname{#1}%
        \expandafter\xdef\csname @author@altaffil@\@roman\acs@author@cnt
          \endcsname{\the\acs@affil@alt@cnt}%
      }%
    \endgroup
  \else
    \acs@warning{Affiliation with no author}%
  \fi
}
\@onlypreamble\altaffiliation
%    \end{macrocode}
%\end{macro}
%\begin{macro}{\acs@altaffil@ifdup}
%\begin{macro}{\acs@altaffil@ifdup@aux}
% This is very similar to the same routine for normal affiliations but
% with the appropriate name changes.
%    \begin{macrocode}
\newcommand*\acs@altaffil@ifdup[1]{%
  \def\@tempa{#1}%
  \@tempswafalse
  \@tempcnta\z@\relax
  \ifnum\acs@affil@alt@cnt>\z@\relax
    \expandafter\acs@altaffil@ifdup@aux
  \fi 
  \if@tempswa
    \expandafter\@firstoftwo
  \else
    \expandafter\@secondoftwo
  \fi
}
\newcommand*\acs@altaffil@ifdup@aux{%
  \advance\@tempcnta\@ne\relax
  \expandafter\expandafter\expandafter\def\expandafter\expandafter
    \expandafter\@tempb\expandafter\expandafter\expandafter
    {\csname @altaffil@\@roman\@tempcnta\endcsname}%
  \ifx\@tempa\@tempb
    \expandafter\@tempswatrue
  \else
    \ifnum\@tempcnta<\acs@affil@alt@cnt\relax
      \expandafter\expandafter\expandafter\acs@altaffil@ifdup@aux
    \fi
  \fi
}
%    \end{macrocode}
%\end{macro}
%\end{macro}
%\begin{macro}{\email}
% E-mail addresses are attached to authors as well.
%    \begin{macrocode}
\newcommand*\email[1]{%
  \ifnum\acs@author@cnt>\z@\relax
    \expandafter\gdef\csname @email@\@roman\acs@author@cnt\endcsname
      {#1}%
  \else
    \acs@warning{E-mail with no author}%
  \fi
}
\@onlypreamble\email
%    \end{macrocode}
%\end{macro}
%\begin{macro}{\fax}
%\begin{macro}{\phone}
% Fax and phone numbers are similar.
%    \begin{macrocode}
\newcommand*\fax[1]{%
  \ifnum\acs@author@cnt>\z@\relax
    \expandafter\gdef\csname @fax@\@roman\acs@author@cnt\endcsname
      {#1}%
  \else
    \acs@warning{Fax number with no author}%
  \fi
}
\@onlypreamble\fax
\newcommand*\phone[1]{%
  \ifnum\acs@author@cnt>\z@\relax
    \expandafter\gdef\csname @phone@\@roman\acs@author@cnt\endcsname
      {#1}%
  \else
    \acs@warning{Phone number with no author}%
  \fi
}
\@onlypreamble\phone
%    \end{macrocode}
%\end{macro}
%\end{macro}
%
%\begin{macro}{\abbreviations}
%\begin{macro}{\@abbreviations}
%\begin{macro}{\keywords}
%\begin{macro}{\@keywords}
% Some journals use these.
%    \begin{macrocode}
\newcommand*\abbreviations[1]{%
  \gdef\@abbreviations{#1}%
}
\newcommand*\@abbreviations{}
\@onlypreamble\abbreviations
\newcommand*\keywords[1]{%
  \gdef\@keywords{#1}%
}
\newcommand*\@keywords{}
\@onlypreamble\keywords
%    \end{macrocode}
%\end{macro}
%\end{macro}
%\end{macro}
%\end{macro}
%
%\begin{macro}{\acs@abbreviations@print}
%\begin{macro}{\acs@keywords@print}
%\begin{macro}{\acs@title@short@print}
% For printing the key simple meta-data.
%    \begin{macrocode}
\newcommand*\acs@abbreviations@print{%
  \ifx\@abbreviations\@empty\else
    \section*{Abbreviations}
    \@abbreviations
    \par
  \fi
}
\newcommand*\acs@keywords@print{%
  \ifx\@keywords\@empty\else
    \section*{Keywords}
    \@keywords
    \par
  \fi
}
\newcommand*\acs@title@short@print{%
  \section*{Running header}
  \acs@title@short
  \par
}
%    \end{macrocode}
%\end{macro}
%\end{macro}
%\end{macro}
%
%\begin{macro}{\acs@space@pre@title}
%\begin{macro}{\acs@space@post@title}
%\begin{macro}{\acs@space@post@author}
%\begin{macro}{\acs@space@post@address}
%\begin{macro}{\acs@space@post@email}
%\begin{macro}{\acs@maketitle@width}
% Lengths for \cs{@maketitle}.
%    \begin{macrocode}
\newlength\acs@space@pre@title
\setlength\acs@space@pre@title{2em}
\newlength\acs@space@post@title
\setlength\acs@space@post@title{1.5em}
\newlength\acs@space@post@author
\setlength\acs@space@post@author{1em}
\newlength\acs@space@post@address
\setlength\acs@space@post@address{1em}
\newlength\acs@space@post@email
\setlength\acs@space@post@email{1.5em}
\newlength\acs@maketitle@width
\setlength\acs@maketitle@width{\textwidth}
%    \end{macrocode}
%\end{macro}
%\end{macro}
%\end{macro}
%\end{macro}
%\end{macro}
%\end{macro}
%\begin{macro}{\affilsize}
%\begin{macro}{\authorsize}
%\begin{macro}{\emailsize}
%\begin{macro}{\titlesize}
% Some simple size commands.
%    \begin{macrocode}
\newcommand*\affilsize{\normalsize}
\newcommand*\authorsize{\large}
\newcommand*\emailsize{\normalsize}
\newcommand*\titlesize{\LARGE}
%    \end{macrocode}
%\end{macro}
%\end{macro}
%\end{macro}
%\end{macro}
%\begin{macro}{\authorfont}
%\begin{macro}{\authorfont}
%\begin{macro}{\emailfont}
%\begin{macro}{\titlefont}
% Font settings for \cs{@maketitle}.
%    \begin{macrocode}
\newcommand*\affilfont{\itshape}
\newcommand*\authorfont{\sffamily}
\newcommand*\emailfont{}
\newcommand*\titlefont{\bfseries\sffamily}
%    \end{macrocode}
%\end{macro}
%\end{macro}
%\end{macro}
%\end{macro}
%\begin{macro}{\ps@acs}
% A shortcut to make page styles.
%    \begin{macrocode}
\newcommand*{\ps@acs}{}
\let\ps@acs\ps@plain
%    \end{macrocode}
%\end{macro}
%\begin{macro}{\@maketitle}
%\begin{macro}{\@maketitle@title@hook}
% With the changes outlined above in place, a new \cs{@maketitle}
% macro is needed.  This is partially a copy of the existing, but
% rather heavily modified.
%    \begin{macrocode}
\def\@maketitle{%
  \pagestyle{acs}%
  \ifnum\acs@author@cnt<\z@\relax
    \acs@warning{No authors defined: At least one author is required}%
  \fi
  \newpage
  \null
  \vspace*{\acs@space@pre@title}%
  \begin{center}
    \begin{minipage}{\acs@maketitle@width}
      \begin{center}
        {%
          \titlefont
          \titlesize
          \let\@fnsymbol\acs@author@fnsymbol
          \let\footnote\acs@title@footnote
          \acs@maketitle@suppinfo \@title
          \acs@title@footnote@check
          \global\acs@footnote@cnt\c@footnote
          \@maketitle@title@hook
          \par
        }%
        \vspace*{\acs@space@post@title}%
        {%
          \authorsize
          \authorfont
          \frenchspacing
          \acs@author@list
          \par
        }%
        \vspace*{\acs@space@post@author}%
        {%
          \affilsize
          \affilfont
          \acs@address@list
          \par
        }%
        \vspace*{\acs@space@post@address}%
        {%
          \emailsize
          \emailfont
          \ifacs@email
            \expandafter\acs@contact@details
          \fi
        }%
        \vspace*{\acs@space@post@email}%
      \end{center}
    \end{minipage}
  \end{center}%
}
\newcommand*\@maketitle@title@hook{}
%    \end{macrocode}
%\end{macro}
%\end{macro}
%\begin{macro}{\acs@maketitle@suppinfo}
% This is spun out so that it can be avoided if necessary: this is done
% on the sly.
%    \begin{macrocode}
\newcommand*\acs@maketitle@suppinfo{%
  \ifx\acs@manuscript\acs@manuscript@suppinfo
    Supporting Information:\\
  \fi
}
%    \end{macrocode}
%\end{macro}
%\begin{macro}{\acs@title@footnote}
%\begin{macro}{\acs@title@footnote@check}
% Footnotes need to be created so that they appear correctly.
%    \begin{macrocode}
\newcommand*\acs@title@footnote[1]{%
  \footnotemark
  \g@addto@macro\@thanks{\footnotetext{#1}}%
}
\newcommand\acs@title@footnote@check{%
  \ifx\@thanks\@empty
  \else
    \begingroup
      \toks@=\expandafter{\@thanks}%
      \xdef\@thanks{%
        \begingroup
          \let\noexpand\@fnsymbol\noexpand\acs@author@fnsymbol
          \the\toks@
        \endgroup
      }%
    \endgroup
  \fi
}
%    \end{macrocode}
%\end{macro}
%\end{macro}
%\begin{macro}{\acs@contact@details}
% A general contact details macro.
%    \begin{macrocode}
\newcommand*\acs@contact@details{%
  { \sffamily E-mail: \acs@email@list }%
  \acs@number@list
}
%    \end{macrocode}
%\end{macro}
%\begin{macro}{\@thanks}
% The \cs{@thanks} macro is used as a hook to generate the footnotes if
% needed.
%    \begin{macrocode}
\let\@thanks\@empty
%    \end{macrocode}
%\end{macro}
%\begin{macro}{\acs@author@list}
% Printing the author list needs to do several things. The appropriate
% separators between authors are created and the author names
% themselves are printed.
%    \begin{macrocode}
\newcommand*\acs@author@list{%
  \@tempcnta\z@\relax
  \ifnum\acs@author@cnt=\z@\relax\else
    \expandafter\acs@author@list@main
  \fi
}
%    \end{macrocode}
%\end{macro}
%\begin{macro}{\acs@author@footnotes}
%\begin{macro}{\acs@author@list@main}
% The main control macro for producing the author list iterates over
% each author on the list. The result is stored as
% \cs{acs@author@listing}.
%    \begin{macrocode}
\newcommand*\acs@author@footnotes{}
\newcommand*\acs@author@list@main{%
  \advance\@tempcnta\@ne\relax
  \def\acs@author@footnotes{}%
  \acs@author@list@and
  \space
  \@nameuse{@author@\@roman\@tempcnta}%
  \acs@author@list@comma
  \acs@author@star
  \acs@author@affil
  \acs@author@affil@also
  \acs@author@affil@alt
  \ifx\@empty\acs@author@footnotes\else
    \textsuperscript{\acs@author@footnotes}%
  \fi
  \ifnum\@tempcnta<\acs@author@cnt\relax
    \expandafter\acs@author@list@main
  \fi
}
%    \end{macrocode}
%\end{macro}
%\end{macro}
%\begin{macro}{\acs@author@list@and}
%\begin{macro}{\acs@author@list@comma}
% Simple checks to add an ``and'' and a comma.
%    \begin{macrocode}
\newcommand*\acs@author@list@and{%
  \ifnum\acs@author@cnt=\@ne\relax\else
    \ifnum\@tempcnta=\acs@author@cnt\relax
      \space and%
    \fi
  \fi
}
\newcommand*\acs@author@list@comma{%
  \ifnum\acs@author@cnt>\tw@\relax
    \ifnum\@tempcnta<\acs@author@cnt\relax
      ,%
    \fi
  \fi
}
%    \end{macrocode}
%\end{macro}
%\end{macro}
%\begin{macro}{\acs@author@star}
%\begin{macro}{\acs@author@star@aux}
% A check for an e-mail for an author: if so, add a star.
%    \begin{macrocode}
\newcommand*\acs@author@star{%
  \acs@ifundefined{@email@\@roman\@tempcnta}{}{%
    \acs@author@star@aux
  }%
}
\newcommand*\acs@author@star@aux{%
  \protected@edef\acs@author@footnotes{%
    \acs@author@fnsymbol{\z@}%
    \ifnum\acs@affil@cnt>\@ne\relax
      ,%
    \else
      \ifnum\acs@affil@alt@cnt>\z@\relax
        ,%
      \fi
    \fi
  }%
}
%    \end{macrocode}
%\end{macro}
%\end{macro}
%\begin{macro}{\acs@author@affil}
%\begin{macro}{\acs@author@affil@aux}
% The main affiliation of the author is checked for, and assuming one
% is found the appropriate symbol is added to the list.
%    \begin{macrocode}
\newcommand*\acs@author@affil{%
  \acs@ifundefined{%
    @affil@\@roman\csname @author@affil@\@roman\@tempcnta\endcsname
  }{%
    \acs@warning{%
      No affiliation given for author\MessageBreak
      \@nameuse{@author@\@roman\@tempcnta}%
    }%
  }{%
    \acs@author@affil@aux
  }%
}
\newcommand*\acs@author@affil@aux{%
  \ifnum\acs@affil@cnt>\@ne\relax
    \expandafter\acs@affil@marker@cnt\csname @author@affil@\@roman
      \@tempcnta\endcsname\relax
    \advance\acs@affil@marker@cnt\acs@footnote@cnt\relax
    \protected@edef\acs@author@footnotes{%
      \acs@author@footnotes
      \acs@author@fnsymbol{\acs@affil@marker@cnt}%
    }%
  \else
    \ifnum\acs@affil@alt@cnt>\z@\relax
      \acs@affil@marker@cnt\@ne\relax
      \advance\acs@affil@marker@cnt\acs@footnote@cnt\relax
      \protected@edef\acs@author@footnotes{%
        \acs@author@footnotes
        \acs@author@fnsymbol{\acs@affil@marker@cnt}%
      }%
    \fi
  \fi
}
%    \end{macrocode}
%\end{macro}
%\end{macro}
%\begin{macro}{\acs@author@affil@also}
%\begin{macro}{\acs@author@affil@also@aux}
% The ``also'' affiliations are generated by a loop as there may be more
% than one.
%    \begin{macrocode}
\newcommand*\acs@author@affil@also{%
  \acs@ifundefined{@author@alsoaffil@\@roman\@tempcnta}{}{%
    \acs@author@affil@also@aux
  }%
}
\newcommand*\acs@author@affil@also@aux{%
  \expandafter\@for\expandafter\@tempa\expandafter:\expandafter
    =\csname @author@alsoaffil@\@roman\@tempcnta\endcsname\do{%
    \acs@affil@marker@cnt\@tempa\relax
    \advance\acs@affil@marker@cnt\acs@footnote@cnt\relax
    \protected@edef\acs@author@footnotes{%
      \acs@author@footnotes
      ,%
      \acs@author@fnsymbol{\acs@affil@marker@cnt}%
    }%
  }%
}
%    \end{macrocode}
%\end{macro}
%\end{macro}
%\begin{macro}{\acs@author@affil@alt}
%\begin{macro}{\acs@author@affil@alt@aux}
% Alternative affiliations get the correct affiliation number back out
% from the stored data. There are then two corrections: one for the
% total number of main affiliations and a second in case there is a
% footnote to the title.
%    \begin{macrocode}
\newcommand*\acs@author@affil@alt{%
  \acs@ifundefined{@author@altaffil@\@roman\@tempcnta}{}{%
    \acs@author@affil@alt@aux
  }%
}
\newcommand*\acs@author@affil@alt@aux{%
  \expandafter\acs@affil@marker@cnt
    \csname @author@altaffil@\@roman\@tempcnta\endcsname\relax
  \advance\acs@affil@marker@cnt\acs@affil@cnt\relax
  \advance\acs@affil@marker@cnt\acs@footnote@cnt\relax
  \protected@edef\acs@author@footnotes{%
    \acs@author@footnotes
    ,%
    \acs@author@fnsymbol{\acs@affil@marker@cnt}%
  }%
}
%    \end{macrocode}
%\end{macro}
%\end{macro}
% \begin{macro}{\acs@author@fnsymbol}
% \begin{macro}{\acs@author@fnsymbol@aux}
% \begin{macro}{\acs@author@fnsymbol@loop}
% \begin{macro}{\acs@author@fnsymbol@loop@aux@i}
% \begin{macro}{\acs@author@fnsymbol@loop@aux@ii}
% \begin{macro}{\acs@author@fnsymbol@loop@aux@m}
% \begin{macro}{\acs@author@fnsymbol@loop@aux@Q}
% \begin{macro}{\acs@author@fnsymbol@symbol}
% The ACS have an extended list of symbols.  The star appears at the
% special position zero.
%    \begin{macrocode}
\newcommand*{\acs@author@fnsymbol}[1]{%
  \ensuremath{%
    \expandafter\acs@author@fnsymbol@aux\expandafter{\number#1 }%
  }%
}
\newcommand*{\acs@author@fnsymbol@aux}[1]{%
  \ifnum#1>10 %
    \expandafter\acs@author@fnsymbol@loop
  \else
    \expandafter\acs@author@fnsymbol@symbol
  \fi
    {#1}%
}
\newcommand*{\acs@author@fnsymbol@loop}[1]{%
   \acs@author@fnsymbol@loop@aux@i#1%
}
\newcommand*{\acs@author@fnsymbol@loop@aux@i}[2]{%
  \acs@author@fnsymbol@symbol{\ifnum#2=0 10\else #2\fi}%
  \expandafter\acs@author@fnsymbol@loop@aux@ii\romannumeral #1000Q{}%
    {\acs@author@fnsymbol@symbol{\ifnum#2=0 10\else #2\fi}}%
}
\newcommand*{\acs@author@fnsymbol@loop@aux@ii}[1]{%
  \@nameuse{acs@author@fnsymbol@loop@aux@#1}%
}
\def\acs@author@fnsymbol@loop@aux@m#1Q#2#3{%
  \acs@author@fnsymbol@loop@aux@ii#1Q{#2#3}{#3}%
}
\newcommand*{\acs@author@fnsymbol@loop@aux@Q}[2]{#1}
\newcommand*{\acs@author@fnsymbol@symbol}[1]{%
  \ifcase #1 *\or
    \dagger\or
      \ddagger\or
        \P\or
          \S\or
            \|\or
              \bot\or
                \#\or
                  @\or
                    \triangle\or
                      \nabla
    \fi
}
%    \end{macrocode}
% \end{macro}
% \end{macro}
% \end{macro}
% \end{macro}
% \end{macro}
% \end{macro}
% \end{macro}
% \end{macro}
% \begin{macro}{\acs@address@list}
% \begin{macro}{\acs@address@list@auxi}
% \begin{macro}{\acs@address@list@auxii}
%   Loop over the addresses and any extra affiliations and print them
%   all: if there is only one, omit the marker entirely. There is also a
%   need to watch out for any footnotes from the title.
%    \begin{macrocode}
\newcommand*\acs@address@list{%
  \ifnum\acs@affil@cnt>\z@
    \expandafter\acs@address@list@auxi
  \else
    \acs@warning{No affiliations: at least one affiliation is needed}%
  \fi
}
\newcommand*\acs@address@list@auxi{%
  \ifnum0%
    \ifnum\acs@affil@cnt>\@ne    1\fi
    \ifnum\acs@affil@alt@cnt>\z@ 1\fi
      >\z@
    \expandafter\acs@address@list@auxii
  \else
    \@address@i\par
  \fi
}
\newcommand*\acs@address@list@auxii{%
  \@tempcnta\z@
  \acs@affil@marker@cnt\acs@footnote@cnt
  \loop\ifnum\@tempcnta<\acs@affil@cnt
    \advance\@tempcnta\@ne
    \advance\acs@affil@marker@cnt\@ne
    \acs@author@fnsymbol{\acs@affil@marker@cnt}%
    \@nameuse{@address@\@roman\@tempcnta}\par
  \repeat
  \@tempcnta\z@
  \loop\ifnum\@tempcnta<\acs@affil@cnt
    \advance\@tempcnta\@ne
    \advance\acs@affil@marker@cnt\@ne
    \acs@ifundefined{@altaffil@\@roman\@tempcnta}
      {}
      {%
        \acs@author@fnsymbol{\acs@affil@marker@cnt}%
        \@nameuse{@altaffil@\@roman\@tempcnta}\par
      }%
  \repeat
}
%    \end{macrocode}
% \end{macro}
% \end{macro}
% \end{macro}
%\begin{macro}{\acs@fnsymbol@org}
%\begin{macro}{\acs@affil@list}
%\begin{macro}{\acs@affil@list@aux}
% Footnotes are done in two stages.  First the main affiliation is
% handled, then the possible alternative. There is a need to check for
% the possibility that there is only one main affiliation but one or
% more alternative ones.
%    \begin{macrocode}
\newcommand*\acs@fnsymbol@org{}
\newcommand*\acs@affil@list{%
  \let\acs@fnsymbol@org\@fnsymbol
  \let\@fnsymbol\acs@author@fnsymbol
  \@tempcnta\z@\relax
  \@tempcntb\z@\relax
  \ifnum\acs@affil@cnt>\@ne\relax
    \expandafter\acs@affil@list@aux
  \else
    \ifnum\acs@affil@alt@cnt>\z@\relax
      \acs@affil@marker@cnt\@ne\relax
      \advance\acs@affil@marker@cnt\acs@footnote@cnt\relax
      \footnotetext[\acs@affil@marker@cnt]{\@affil@i}%
      \@tempcnta\@ne\relax
    \fi
  \fi
  \ifnum\acs@affil@alt@cnt>\z@\relax
    \expandafter\acs@affil@alt@list
  \fi
  \let\@fnsymbol\acs@fnsymbol@org
}
\newcommand*\acs@affil@list@aux{%
  \advance\@tempcnta\@ne\relax
  \acs@affil@marker@cnt\@tempcnta\relax
  \advance\acs@affil@marker@cnt\acs@footnote@cnt\relax
  \footnotetext[\acs@affil@marker@cnt]{%
    \@nameuse{@affil@\@roman\@tempcnta}%
  }%
  \ifnum\@tempcnta<\acs@affil@cnt\relax
    \expandafter\acs@affil@list@aux
  \fi
}
%    \end{macrocode}
%\end{macro}
%\end{macro}
%\end{macro}
%\begin{macro}{\acs@affil@alt@list}
%\begin{macro}{\acs@affil@alt@lista@aux}
% The secondary loop for alternative affiliations is similar.
%    \begin{macrocode}
\newcommand*\acs@affil@alt@list{%
  \advance\@tempcntb\@ne\relax
  \acs@ifundefined{@altaffil@\@roman\@tempcntb}{}{%
    \acs@altaffil@foot@aux
  }
  \ifnum\@tempcntb<\acs@author@cnt\relax
    \expandafter\acs@affil@alt@list
  \fi
}
\newcommand*\acs@altaffil@foot@aux{%
  \advance\@tempcnta\@ne\relax
  \acs@affil@marker@cnt\@tempcnta\relax
  \advance\acs@affil@marker@cnt\acs@footnote@cnt\relax
  \footnotetext[\acs@affil@marker@cnt]{%
    \@nameuse{@altaffil@\@roman\@tempcntb}%
  }%
}
%    \end{macrocode}
%\end{macro}
%\end{macro}
%\begin{macro}{\acs@email@list@font}
%\begin{macro}{\acs@email@list}
%\begin{macro}{\acs@email@list@aux}
% The final piece of meta-data to print is the e-mail address list.
% The total number of e-mail addresses given it counted in
% \cs{@tempcntb}, which means a warning can be given if there are
% none.  The group is used so that \cs{UrlFont} can be set correctly.
%    \begin{macrocode}
\newcommand*\acs@email@list@font{\sf}
\newcommand*\acs@email@list{%
  \begingroup
    \def\UrlFont{\acs@email@list@font}%
    \@tempcnta\z@\relax
    \@tempcntb\z@\relax
    \acs@email@list@aux
    \ifnum\@tempcntb=\z@\relax
      \acs@warning{%
        No e-mail given:\MessageBreak
        at lest one author must have a contact e-mail%
      }%
    \fi
  \endgroup
}
\newcommand*\acs@email@list@aux{%
  \advance\@tempcnta\@ne\relax
  \ifnum\@tempcnta>\acs@author@cnt\relax\else
    \acs@ifundefined{@email@\@roman\@tempcnta}{}{%
      \advance\@tempcntb\@ne\relax
      \ifnum\@tempcntb>\@ne\relax
        ;
      \fi
      \expandafter\expandafter\expandafter\url\expandafter
        \expandafter\expandafter{%
           \csname @email@\@roman\@tempcnta\endcsname
      }%
    }%
    \expandafter\acs@email@list@aux
  \fi
}
%    \end{macrocode}
%\end{macro}
%\end{macro}
%\end{macro}
%\begin{macro}{\acs@number@list}
%\begin{macro}{\acs@number@list@aux@i}
%\begin{macro}{\acs@number@list@aux@ii}
% Listing phone and fax numbers is easier as they don't have to be
% given. Everything is done in one block so that it is possible to know
% whether to add a new line and also to keep everything together.
%    \begin{macrocode}
\newcommand*\acs@number@list{%
  \begingroup
    \acs@number@list@aux@i{phone}%
    \let\@tempb\@tempa
    \acs@number@list@aux@i{fax}%  
    \ifx\@tempa\@empty
      \let\@tempa\@tempb
    \else
      \ifx\@tempb\@empty\else
        \protected@edef\@tempa{%
          \@tempb.\space\@tempa
        }%
      \fi
    \fi
    \ifx\@tempa\@empty\else
      \par
      \@tempa
    \fi
  \endgroup
}
\newcommand*\acs@number@list@aux@i[1]{%
  \def\@tempa{}%
  \@tempcnta\z@\relax
  \def\acs@number@list@aux@ii{%
    \advance\@tempcnta\@ne\relax
    \ifnum\@tempcnta>\acs@author@cnt\relax\else
      \acs@ifundefined{@#1@\@roman\@tempcnta}{}{%
        \acs@ifundefined{@email@\@roman\@tempcnta}{}{%
          \ifx\@tempa\@empty
            \edef\@tempa{%
              \@nameuse{@#1@\@roman\@tempcnta}%
            }%
          \else
            \edef\@tempa{%
              \@tempa
              ;
              \@nameuse{@#1@\@roman\@tempcnta}%
            }%
          \fi
        }%
      }%
      \expandafter\acs@number@list@aux@ii
    \fi
  }%
  \acs@number@list@aux@ii
  \ifx\@tempa\@empty\else
    \protected@edef\@tempa{%
      \MakeUppercase#1: \@tempa
    }%
  \fi
}
\newcommand*\acs@number@list@aux@ii{}
%    \end{macrocode}
%\end{macro}
%\end{macro}
%\end{macro}
% \begin{macro}{\endabstract}
% \begin{macro}{\acs@abstract@extras}
%    \begin{macrocode}
\g@addto@macro\endabstract{%
  \aftergroup\acs@abstract@extras
}
\newcommand*{\acs@abstract@extras}{%
  \ifacs@abbreviations
    \acs@abbreviations@print
    \par
  \fi
  \ifacs@keywords
    \acs@keywords@print
    \par
  \fi
}
%    \end{macrocode}
% \end{macro}
% \end{macro}
%
%\begin{macro}{\acs@maketitle@extras}
%\begin{macro}{\acs@maketitle@extras@hook}
% A couple of things might need to be added to \cs{maketitle}.
%    \begin{macrocode}
\newcommand*\acs@maketitle@extras{%
  \acs@maketitle@extras@hook
}
\newcommand*\acs@maketitle@extras@hook{}
\g@addto@macro{\maketitle}{\acs@maketitle@extras}
%    \end{macrocode}
%\end{macro}
%\end{macro}
% \cs{maketitle} is required by the document class, and must start
% the document.  No variation is allowed, and so it is done
% automatically.
%    \begin{macrocode}
\g@addto@macro{\document}{\maketitle}
%    \end{macrocode}
%   
%\begin{environment}{scheme}
%\begin{environment}{chart}
%\begin{environment}{graph}
% Three new float types are provided, \texttt{scheme}, \texttt{chart}
% and \texttt{graph}.  These are the most obvious types; for graphs,
% a slight problem arises with the file extension.
%    \begin{macrocode}
\newfloat{scheme}{htbp}{los}
\floatname{scheme}{Scheme}
\newfloat{chart}{htbp}{loc}
\floatname{chart}{Chart}
\newfloat{graph}{htbp}{loh}
\floatname{graph}{Graph}
%    \end{macrocode}
%\end{environment}
%\end{environment}
%\end{environment}
%\begin{macro}{\schemename}
%\begin{macro}{\chartname}
%\begin{macro}{\graphname}
% Naming is set up in the same way as the kernel floats.
%    \begin{macrocode}
\newcommand*\schemename{Scheme}
\newcommand*\chartname{Chart}
\newcommand*\graphname{Graph}
%    \end{macrocode}
%\end{macro}
%\end{macro}
%\end{macro}
% The standard floats should appear ``here'' by default.
%    \begin{macrocode}
\floatplacement{table}{htbp}
\floatplacement{figure}{htbp}
\floatstyle{plaintop}
\restylefloat{table}
%    \end{macrocode}
%\begin{macro}{\acs@floatboxreset}
% Floats are all centred.
%    \begin{macrocode}
\let\acs@floatboxreset\@floatboxreset
\def\@floatboxreset{%
  \centering
  \acs@floatboxreset
}
%    \end{macrocode}
%\end{macro}
%\begin{macro}{\plainref}
%\begin{macro}{\ref}
% For legacy support.
%    \begin{macrocode}
\newcommand*\plainref{}
\AtBeginDocument{\let\plainref\ref}
%    \end{macrocode}
%\end{macro}
%\end{macro}
%   
%\begin{macro}{\acs@section}
%\begin{macro}{\acs@subsection}
%\begin{macro}{\acs@subsubsection}
%\begin{macro}{\acs@startsection@orig}
% Both the numbering and existence of section headers may need to be
% altered. Some generic functions are therefore provided to deal with
% this cleanly. First, some original definitions are saved.
%    \begin{macrocode}
\newcommand*\acs@section{}
\let\acs@section\section
\newcommand*\acs@subsection{}
\let\acs@subsection\subsection
\newcommand*\acs@subsubsection{}
\let\acs@subsubsection\subsubsection
\newcommand*\acs@startsection@orig{}
\let\acs@startsection@orig\@startsection
%    \end{macrocode}
%\end{macro}
%\end{macro}  
%\end{macro}
%\end{macro}
%\begin{macro}{\acs@startsection}
%  A version of \cs{@startsection} which adds unnumbered sections to
%  the TOC: modelled on \pkg{amsart}. This is active as standard.
%    \begin{macrocode}
\newcommand\acs@startsection[6]{%
 \if@noskipsec \leavevmode \fi
 \par \@tempskipa #4\relax
 \@afterindenttrue
 \ifdim \@tempskipa <\z@ \@tempskipa -\@tempskipa \@afterindentfalse\fi
 \if@nobreak \everypar{}\else
     \addpenalty\@secpenalty\addvspace\@tempskipa\fi
 \@ifstar{\@dblarg{\@sect{#1}{\@m}{#3}{#4}{#5}{#6}}}%
         {\@dblarg{\@sect{#1}{#2}{#3}{#4}{#5}{#6}}}%
}
\let\@startsection\acs@startsection
%    \end{macrocode}
%\end{macro}
%\begin{macro}{\acs@startsection@alt}
% An alternative version of \cs{@startsection} which never adds numbers.
%    \begin{macrocode}
\newcommand*\acs@startsection@alt[6]{%
  \if@noskipsec \leavevmode \fi
  \par \@tempskipa #4\relax
  \@afterindenttrue
 \ifdim \@tempskipa <\z@ \@tempskipa -\@tempskipa \@afterindentfalse\fi
 \if@nobreak \everypar{}\else
     \addpenalty\@secpenalty\addvspace\@tempskipa\fi
  \@ifstar{\@ssect{#3}{#4}{#5}{#6}}
          {\@ssect{#3}{#4}{#5}{#6}}%
}
%    \end{macrocode}
%\end{macro}
%\begin{macro}{\acs@sections@none}
%\begin{macro}{\acs@sections@none@aux}
% When removing sections entirely, a gobble macro is needed.
%    \begin{macrocode}
\newcommand*\acs@sections@none{%
  \@ifstar{%
    \acs@sections@none@aux
  }{%
    \acs@sections@none@aux
  }%
}
\newcommand*\acs@sections@none@aux[2][]{%
  \acs@warning{%
    (Sub)section `#2' ignored%
  }%
}
%    \end{macrocode}
%\end{macro}
%\end{macro}
%\begin{macro}{\SectionNumbersOff}
%\begin{macro}{\SectionNumbersOn}
% To macros to add or remove the section numbers. The standard setting
% for the class has them on, but some configurations will turn them
% off. The names of these functions are both in design space so that
% users can change the decision easily.
%    \begin{macrocode}
\newcommand*\SectionNumbersOff{%
  \let\@startsection\acs@startsection@alt
}
\@onlypreamble\SectionNumbersOff
\newcommand*\SectionNumbersOn{%
  \let\@startsection\acs@startsection
}
\@onlypreamble\SectionNumbersOn
%    \end{macrocode}
%\end{macro}
%\end{macro}
%\begin{macro}{\SectionsOff}
%\begin{macro}{\SectionsOn}
% Quite similar for entire sections.
%    \begin{macrocode}
\newcommand*\SectionsOff{%
  \let\section\acs@sections@none
  \let\subsection\acs@sections@none
  \let\subsubsection\acs@sections@none
}
\@onlypreamble\SectionsOff
\newcommand*\SectionsOn{%
  \let\section\acs@section
  \let\subsection\acs@subsection
  \let\subsubsection\acs@subsubsection
}
\@onlypreamble\SectionsOn
%    \end{macrocode}
%\end{macro}
%\end{macro}
% \begin{macro}{\tableofcontents}
%   Never print TOC in itself.
%    \begin{macrocode}
\begingroup
  \toks@=\expandafter{\tableofcontents}
  \xdef\tableofcontents{%
    \begingroup
      \let\noexpand\@startsection\noexpand\acs@startsection@orig
      \the\toks@
    \endgroup
  }
\endgroup
%    \end{macrocode}
% \end{macro}
%\begin{environment}{acknowledgement}
%\begin{environment}{suppinfo}
% Simple named sections.
%    \begin{macrocode}
\newenvironment{acknowledgement}{%
  \acs@section*{\acknowledgementname}%
}{}
\newenvironment{suppinfo}{%
  \acs@section*{\suppinfoname}%
}{}
%    \end{macrocode}
%\end{environment}
%\end{environment}
%\begin{macro}{\acknowledgementname}
%\begin{macro}{\bibsection}
%\begin{macro}{\suppinfoname}
% A few macros need to get around the changes.
%    \begin{macrocode}
\newcommand*\acknowledgementname{Acknowledgement}
\AtEndOfClass{%
  \def\bibsection{%
    \acs@section*{\refname}%
  }%
}
\newcommand*\suppinfoname{Supporting Information Available}
%    \end{macrocode}
%\end{macro}
%\end{macro}
%\end{macro}
%
%\begin{macro}{\acs@abstract}
%\begin{macro}{\acs@endabstract}
%\begin{macro}{\acs@abstract@start}
%\begin{macro}{\acs@abstract@end}
%\begin{macro}{\acs@abstract@iffalse}
% Removing the abstract, if necessary, is done using a trick from the
% \pkg{comment} package. However, it code is copied here to keep
% requirements down.
%    \begin{macrocode}
\newcommand*\acs@abstract{}
\let\acs@abstract\abstract
\newcommand*\acs@endabstract{}
\let\acs@endabstract\endabstract
\begingroup
  \catcode`{ \active
  \catcode`} 12\relax
  \catcode`( 1\relax
  \catcode`) 2\relax
  \gdef\acs@abstract@start(%
    \acs@warning(%
      Abstract not allowed for this\MessageBreak
       manuscript type
    )%
    \@bsphack
    \catcode`{ \active
    \catcode`} 12\relax
    \let\end\fi
    \let{\acs@abstract@end% }
    \iffalse
  )%{
  \gdef\acs@abstract@end#1}(%
    \def\@tempa(#1)%
    \ifx\@tempa\@currenvir
      \@Esphack\endgroup
      \if@ignore
        \global\@ignorefalse
        \ignorespaces
      \fi
    \else
       \expandafter\acs@abstract@iffalse
    \fi
  )
\endgroup
\newcommand*\acs@iffalse{\iffalse}
%    \end{macrocode}
%\end{macro}
%\end{macro}
%\end{macro}
%\end{macro}
%\end{macro}
%\begin{macro}{\AbstractOff}
%\begin{macro}{\AbstractOn}
% A very similar pattern to before.
%    \begin{macrocode}
\newcommand*\AbstractOff{%
  \let\abstract\acs@abstract@start
  \let\endabstract\acs@abstract@end
}
\@onlypreamble\AbstractOff
\newcommand*\AbstractOn{%
  \let\abstract\acs@abstract
  \let\endabstract\acs@endabstract
}
\@onlypreamble\AbstractOn
%    \end{macrocode}
%\end{macro}
%\end{macro}
%   
%\begin{macro}{\acs@collect@toks}
%\begin{macro}{\acs@collect@content}
%\begin{macro}{\acs@collect@content}
% The content of the graphic TOC entry is processed using a method from
% \pkg{amsmath} \emph{via} \pkg{environ}. The entire environment is
% gathered for typesetting in a box. First, some storage is needed.
%    \begin{macrocode}
\newtoks\acs@collect@toks
\newtoks\acs@collect@empty@toks
\newcommand*\acs@collect@begins{}
\newcommand*\acs@collect@content{}
%    \end{macrocode}
%\end{macro}
%\end{macro}
%\end{macro}
%\begin{macro}{\acs@collect}
%\begin{macro}{\acs@collect@aux}
%\begin{macro}{\acs@collect@begins@}
%\begin{macro}{\acs@collect@body}
% This is a \cs{long} version of \cs{collect@body}.
%    \begin{macrocode}
\newcommand\acs@collect[1]{%
  \acs@collect@toks{%
    \expandafter#1\expandafter{\the\acs@collect@toks}%
  }%
  \edef\acs@collect@content{%
    \the\acs@collect@toks
    \noexpand\end{\@currenvir}%
  }%
  \acs@collect@toks\acs@collect@empty@toks
  \def\acs@collect@begins{b}%
  \begingroup
    \expandafter\let\csname\@currenvir\endcsname\acs@collect@aux
    \edef\acs@collect@content{%
      \expandafter\noexpand\csname\@currenvir\endcsname
    }%
    \acs@collect@content
}
\newcommand*\acs@collect@aux{}
\long\def\acs@collect@aux#1\end#2{%
  \edef\acs@collect@begins{%
    \acs@collect@begins@#1\begin\end
    \expandafter\@gobble\acs@collect@begins
  }%
  \ifx\@empty\acs@collect@begins
    \endgroup
    \@checkend{#2}%
    \acs@collect@body{#1}%
  \else
    \acs@collect@body{#1\end{#2}}%
  \fi
  \acs@collect@content
}
\newcommand*\acs@collect@begins@{}
\long\def\acs@collect@begins@#1\begin#2{%
  \ifx\end#2\else
    b\expandafter\acs@collect@begins@
  \fi
}
\newcommand\acs@collect@body[1]{%
  \global\acs@collect@toks\expandafter{\the\acs@collect@toks#1}%
}
%    \end{macrocode}
%\end{macro}
%\end{macro}
%\end{macro}
%\end{macro}
%\begin{macro}{\acs@abstract@print}
% Delayed abstract printing works in a similar way, but with some
% formatting `built-in'.
%    \begin{macrocode}
\newcommand\acs@abstract@print[1]{%
  \global\long\def\acs@abstract@text{%
    \if@twocolumn
      \@restonecoltrue\onecolumn
    \else
      \@restonecolfalse\newpage
    \fi
    \acs@section*{Abstract}%
    #1%
    \if@restonecol
      \twocolumn
    \else
      \newpage
    \fi
  }%
  \AtEndDocument{\acs@abstract@text}%
}
%    \end{macrocode}
%\end{macro}
% \begin{macro}{\acs@tocentry@print}
% \begin{macro}{\acs@tocentry@print@aux}
% \begin{macro}{\acs@tocentry@text}
% \changes{v3.10b}{2015/06/24}{Define \cs{acs@tocentry@text} in all cases}
% \changes{v3.11a}{2018/04/02}{Print abstract at full size}
% The same approach is taken for the graphical table of content
% printing. This is done in a box so that everything has a frame around
% it.
%    \begin{macrocode}
\newcommand{\acs@tocentry@print}[1]{%
  \gdef\acs@tocentry@text{\normalsize#1}%
  \AtEndDocument{%
    \if@twocolumn
      \@restonecoltrue\onecolumn
    \else
      \@restonecolfalse\newpage
    \fi
    \acs@tocentry@print@aux
    \if@restonecol
      \twocolumn
    \else
      \newpage
    \fi
  }%
}
\newcommand*{\acs@tocentry@print@aux}{%
  \begingroup
    \let\@startsection\acs@startsection@orig
    \acs@section*{\tocentryname}%
    \tocsize
    \sffamily
    \singlespacing
    \begin{center}
      \fbox
        {%
          \begin{minipage}{\acs@tocentry@height}
            \vbox to \acs@tocentry@width{\acs@tocentry@text}%
          \end{minipage}%
        }%
    \end{center}%
  \endgroup
}
\newcommand*\acs@tocentry@text{TOC ENTRY REQUIRED}
\newlength{\acs@tocentry@height}
\newlength{\acs@tocentry@width}
\setlength{\acs@tocentry@height}{9 cm}
\setlength{\acs@tocentry@width}{3.5 cm}
%    \end{macrocode}
% \end{macro}
% \end{macro}
% \end{macro}
%\begin{environment}{tocentry}
% Actually creating the entry is pretty easy.
%    \begin{macrocode}
\newenvironment{tocentry}{\acs@collect\acs@tocentry@print}{}
%    \end{macrocode}
%\end{environment}
%\begin{macro}{\tocentryname}
% A simple name macro.
%    \begin{macrocode}
\newcommand*\tocentryname{Graphical TOC Entry}
%    \end{macrocode}
%\end{macro}
%\begin{macro}{\tocsize}
% The font size for printing the TOC entry.
%    \begin{macrocode}
\newcommand*\tocsize{%
  \@setfontsize\tocsize\@viiipt\@ixpt
}
%    \end{macrocode}
%\end{macro}
%   
%\begin{macro}{\acs@type@list}
%\begin{macro}{\acs@type@default}
%\begin{macro}{\acs@type@check}
% Different journals allow different types of article. A list is set
% up here: different journals can then alter it. A check function
% is also provided along with a default.
%    \begin{macrocode}
\newcommand*\acs@type@list{article,communication,suppinfo}
\newcommand*\acs@type@default{article}
\newcommand*\acs@type@check{%
  \@tempswafalse
  \@for\@tempa:=\acs@type@list\do{%
    \ifx\@tempa\acs@manuscript
      \expandafter\@tempswatrue
    \fi
  }%
  \if@tempswa\else
    \acs@warning{%
      Invalid manuscript type \acs@manuscript:\MessageBreak
      changed to default type \acs@type@default
    }%
    \let\acs@manuscript\acs@type@default
  \fi
}
%    \end{macrocode}
%\end{macro}
%\end{macro}
%\end{macro}
%   
% A few bits for older versions.
%    \begin{macrocode}
\newcommand*\acs@setkeys{\setkeys{acs}}
\let\acs@killabstract\AbstractOff
\let\acs@killsecs\SectionsOff
\newcommand*\acs@validtype[2][article]{%
  \def\acs@type@default{#1}%
  \def\acs@type@list{#2}%
}
%    \end{macrocode}
% 
%\begin{macro}{\acs@par}
% A saved paragraph.
%    \begin{macrocode}
\newcommand*\acs@par{}
\let\acs@par\par
%    \end{macrocode}
%\end{macro}
%\begin{macro}{\acs@layout@shared}
%\begin{environment}{acknowledgement}
%\begin{environment}{suppinfo}
% Some code is used generally when setting up ``press ready'' layouts.
% There is quite a bit here, mainly layout related.
%    \begin{macrocode}
\newcommand*\acs@layout@shared{%
  \AtBeginDocument{\singlespacing}%
  \twocolumn
  \tolerance=2000\relax
  \emergencystretch=10pt\relax
  \geometry{
    letterpaper,
    top    = 12.7mm,
    bottom = 16.8mm,
    left   = 19.3mm,
    right  = 19.3mm
  }%
  \setlength{\columnsep}{8.1mm}%
  \setlength{\parindent}{3.3mm}%
  \renewenvironment{acknowledgement}{%
    \def\@tempa{acknowledgement}%
    \ifx\@currenvir\@tempa
      \let\par\relax
      \acksize
      \vspace{6pt}%
      \textbf{\acknowledgementname}%
    \else
      \acs@section*{\acknowledgementname}%
    \fi
  }{%
    \acs@par
  }%
}
%    \end{macrocode}
%\end{environment}
%\end{environment}
%\end{macro}
%\begin{macro}{\acksize}
%\begin{macro}{\suppsize}
% More sizes.
%    \begin{macrocode}
\newcommand*\acksize{\normalsize}
\newcommand*\suppsize{\normalsize}
%    \end{macrocode}
%\end{macro}
%\end{macro}
%\begin{macro}{\acs@layout@nine}
%\begin{macro}{\@xipt}
%\begin{macro}{\acs@layout@ten}
% The class loads twelve point text. To reset it for print layouts, it
% is easiest to do things directly.
%    \begin{macrocode}
\newcommand*\acs@layout@nine{%
  \def\@xipt{11}%
  \long\def\normalsize{%
    \@setfontsize\normalsize\@ixpt\@xipt
  }%
  \normalsize
  \let\@listi\@listI
  \abovedisplayskip 5\p@ \@plus2\p@ \@minus 5\p@\relax
  \abovedisplayshortskip \z@ \@plus3\p@\relax
  \belowdisplayshortskip 3\p@ \@plus3\p@ \@minus 3\p@\relax
  \belowdisplayskip\abovedisplayskip\relax
  \abovecaptionskip 5\p@\relax
  \intextsep 7\p@ \@plus 2\p@ \@minus 2\p@\relax
}
\newcommand*\acs@layout@ten{%
  \long\def\normalsize{%
    \@setfontsize\normalsize\@xpt\@xiipt
  }%
  \normalsize
  \@setfontsize\normalsize\@xpt\@xiipt
  \let\@listi\@listI
  \abovedisplayskip 10\p@ \@plus2\p@ \@minus5\p@\relax
  \abovedisplayshortskip \z@ \@plus3\p@\relax
  \belowdisplayshortskip 6\p@ \@plus3\p@ \@minus3\p@\relax
  \belowdisplayskip \abovedisplayskip\relax
}
%    \end{macrocode}
%\end{macro}
%\end{macro}
%\end{macro}
%   
% With all of the standard settings done, the journal configuration can
% be loaded.   
%    \begin{macrocode}
\InputIfFileExists{achemso-\acs@journal.cfg}{}{%
  \acs@warning{%
    Unknown journal `\acs@journal':\MessageBreak
    using default configuration JACSAT%
  }%
  \input{achemso-jacsat.cfg}%
}
%    \end{macrocode}
% 
%\begin{macro}{\thepage}     
% Some changes that can always be applied if the manuscript type is
% appropriate: this saves some repetition in the configuration files.
%    \begin{macrocode}
\acs@type@check
\ifx\acs@manuscript\acs@manuscript@note
  \SectionsOff
\fi
\ifx\acs@manuscript\acs@manuscript@review
  \SectionsOn
  \SectionNumbersOn
\fi
\ifx\acs@manuscript\acs@manuscript@suppinfo
  \setkeys{acs}{maxauthors = 0}
  \def\thepage{S-\arabic{page}}
  \renewcommand*\thefigure{S\@arabic\c@figure}
  \renewcommand*\thescheme{S\@arabic\c@scheme}
  \renewcommand*\thetable{S\@arabic\c@table}
  \AtBeginDocument
    {%
      \renewcommand*\citenumfont[1]{S#1}%
      \renewcommand*\bibnumfmt[1]{(S#1)}%
    }
\fi
%    \end{macrocode}  
%\end{macro}
%
% Set up two column layout.
%    \begin{macrocode}
\begingroup
  \def\@tempa{twocolumn}
  \ifx\acs@layout\@tempa
    \aftergroup\acs@layout@shared
  \fi
  \def\@tempa{two-column}
  \ifx\acs@layout\@tempa
    \aftergroup\acs@layout@shared
  \fi
\endgroup
%</class>
%    \end{macrocode}
%
%\subsection{Late shared code}
%
%\begin{macro}{\citenumfont}
% \changes{v3.9b}{2015/03/22}{Allow for setting \opt{super} after
%   loading}
% Changes to citations can now be made. The citation styles supplied
% here require \pkg{natbib}, which is loaded with the appropriate
% options. This part applies to the package as well as the class: it is
% here so that the load order is correct.
%    \begin{macrocode}
%<*class|package>
\ifacs@super
  \RequirePackage[sort&compress,numbers,super]{natbib}
\else
  \RequirePackage[sort&compress,numbers,round]{natbib}
  \def\citenumfont{\textit}
\fi
\define@key{acs}{super}[true]{%
  \def\@tempa{#1}%
  \def\@tempb{true}%
  \ifx\@tempa\@tempb
    \setcitestyle{super,open={},close={}}%
    \renewcommand*\citenumfont{}%
  \else
    \setcitestyle{round}%
    \renewcommand*\citenumfont{\textit}
  \fi
}
\RequirePackage{natmove}
%    \end{macrocode} 
%\end{macro}
% The \pkg{mciteplus} package allows the construction of lists of
% references with sub-letters. However, it might not be available, and
% so it is only loaded if available: the \file{.bst} files should work
% either way. There is also a patch to get cross-references correct
% with the modified \cs{ref} macro used here. The reason for ensuring
% that \cs{@mciteNatbibCiteCmdList} is defined is that it allows for
% the case where people prevent \pkg{mciteplus} being loaded using
% \cs{ver@mciteplus.sty}.
%    \begin{macrocode}
\IfFileExists{mciteplus.sty}{%
  \RequirePackage{mciteplus}
  \providecommand*{\@mciteNatbibCiteCmdList}{}%
  \edef\@tempa{\noexpand\in@{citenum}{\@mciteNatbibCiteCmdList}}%
  \@tempa
  \ifin@
  \else
    \edef\@mciteNatbibCiteCmdList{\@mciteNatbibCiteCmdList,citenum}%
  \fi
%<*!package>
  \def\@mciteSubRef[##1]##2{\plainref{\@mcitereflabelprefix:##1:##2}}
%</!package>
}{}
%    \end{macrocode}
%
%\begin{macro}{\acs@bibstyle}
% The next step is to sort out bibliography formatting. With both the
% package and the class, the bibliography style is determined without
% user intervention. The style is stored as it may need to be
% altered later.
%    \begin{macrocode}
\newcommand*\acs@bibstyle{achemso}
\ifacs@biochem
  \def\acs@bibstyle{biochem}
  \bibliographystyle{biochem}
\else
  \bibliographystyle{achemso}
\fi
%    \end{macrocode}
%\end{macro}
%\begin{macro}{\acs@bibliographystyle}
%\begin{macro}{\bibliographystyle}
% The original \cs{bibliographystyle} macro is now disabled: the
% underlying command is repeated otherwise problems cop up with
% \pkg{chapterbib}.
%    \begin{macrocode}
\newcommand*\acs@bibliographystyle{}
\let\acs@bibliographystyle\bibliographystyle
\def\bibliographystyle#1{%
  \acs@warning{\string\bibliographystyle\space ignored}%
  \expandafter\acs@bibliographystyle\expandafter{\acs@bibstyle}%
}
%</class|package>
%    \end{macrocode}
%\end{macro}
%\end{macro}
%   
%\subsection{Late package-only code}
%
% The \pkg{notes2bib} code may or may not be activated.
%    \begin{macrocode}
%<*package>
\AtBeginDocument{%
  \@ifpackageloaded{notes2bib}{}{%
    \acs@niib@create
  }%
}
%</package>
%    \end{macrocode}
%   
%\subsection{Moving citations with \pkg{natbib}}
%
% The code for moving citations is created as a separate package, as
% the code needed is the same in both cases.
%    \begin{macrocode}
%<*natmove>
\ProvidesPackage{natmove}
  [2010/01/15 v1.1a Automatic citation moving with natbib]
\RequirePackage{natbib}
%    \end{macrocode}
%
%\begin{macro}{\nmv@ifmtarg}
%\begin{macro}{\nmv@xifmtarg}
% To keep down dependence on other packages, the very short code block
% from \pkg{ifmtarg} is copied here with an internal name.
%    \begin{macrocode}
\newcommand*\nmv@ifmtarg{}
\newcommand*\nmv@xifmtarg{}
\begingroup
  \catcode`\Q 3\relax
  \long\gdef\nmv@ifmtarg#1{%
    \nmv@xifmtarg#1QQ\@secondoftwo\@firstoftwo\@nil
  }
  \long\gdef\nmv@xifmtarg#1#2Q#3#4#5\@nil{#4}
\endgroup
%    \end{macrocode}
%\end{macro}
%\end{macro}
%\begin{macro}{\ifnmv@cite}
% A flag is need to watch whether \cs{cite} or another macro is in use.
%    \begin{macrocode}
\newif\ifnmv@cite
%    \end{macrocode}
%\end{macro}
%\begin{macro}{\nmv@citex}
% Using the flag, either the new internal macro, or the \pkg{natbib}
% original, can be called.
%    \begin{macrocode}
\newcommand*\nmv@citex{%
  \ifnmv@cite
    \expandafter\nmv@citex@moving
  \else
    \expandafter\nmv@citex@nat
  \fi
}
%    \end{macrocode}
%\end{macro}
%\begin{macro}{\nmv@after}
% Later, the argument for \cs{cite} will need to be saved.
%    \begin{macrocode}
\newcommand*\nmv@after{}
%    \end{macrocode}
%\end{macro}
%\begin{macro}{\nmv@citex@moving}
% The new version of \cs{@citex} is needed that looks ahead of the
% citation using \cs{futurelet}.  There are three arguments to
% \cs{@citex} when using \pkg{natbib}.  Other than that, the trick
% used here is similar to that in \pkg{cite}.
%    \begin{macrocode}
\newcommand*\nmv@citex@moving{}
\def\nmv@citex@moving[#1][#2]#3{%
  \leavevmode
  \skip@\lastskip\relax
  \unskip
  \begingroup
    \def\nmv@after{\nmv@citex@nat[#1][#2]{#3}}%
    \global\nmv@citefalse
    \nmv@citex@get@next\relax
}
%    \end{macrocode}
%\end{macro}
%\begin{macro}{\nmv@citex@end}
% To get things right at the end.
%    \begin{macrocode}
\newcommand*\nmv@citex@end{%
    \nmv@after
  \endgroup
}
%    \end{macrocode}
%\end{macro}
%\begin{macro}{\nmv@citex@get@next}
%\begin{macro}{\nmv@citex@next}
% The next token on the input stack is saved into
% \cs{nmv@citex@next}, after gobbling up one token.
%    \begin{macrocode}
\newcommand*\nmv@citex@get@next[1]{%
  \futurelet\nmv@citex@next\nmv@citex@punct
}
%    \end{macrocode}
%\end{macro}
%\end{macro}
%\begin{macro}{\nmv@citex@punct}
%\begin{macro}{\nmv@citex@punct@undouble}
%\begin{macro}{\nmv@citex@loop}
% The working macro for moving the punctuation.  This is very much
% like \cs{@citey} in the \pkg{cite} package. The initial assumption
% is that the loop will terminate, and so the recursion call will
% simply do the finalisation.
%    \begin{macrocode}
\newcommand*\nmv@citex@punct{%
  \let\nmv@citex@loop\nmv@citex@end
  \ifx\nmv@citex@next.\relax
    \ifnum\spacefactor<\nmv@citex@sfac\else
      \expandafter\expandafter\expandafter\nmv@citex@punct@undouble
    \fi
  \fi
  \expandafter\nmv@citex@punct@aux\natmovechars\@nil
  \nmv@citex@loop
}
\newcommand*\nmv@citex@punct@undouble{%
  \let\nmv@citex@next\relax
  \let\nmv@citex@loop\nmv@citex@get@next
}
\newcommand*\nmv@citex@loop{}
%    \end{macrocode}
%\end{macro}
%\end{macro}
%\end{macro}
%\begin{macro}{\nmv@citex@punct@aux}
% The final part of the punctuation moving system.
%    \begin{macrocode}
\newcommand*\nmv@citex@punct@aux[1]{%
  \ifx\nmv@citex@next#1\@empty
    #1%
    \let\nmv@citex@loop\nmv@citex@get@next
  \fi
  \ifx#1\@nil\else
    \expandafter\nmv@citex@punct@aux
  \fi
}
%    \end{macrocode}
%\end{macro}
%\begin{macro}{\nmv@citex@sfac}
% The value of the spacing factor after a full stop is used to signal
% doubled punctuation.  For French spacing, a bit of patching is
% needed.
%    \begin{macrocode}
\mathchardef\nmv@citex@sfac3000\relax
\expandafter\def\expandafter\frenchspacing\expandafter{%
  \frenchspacing
  \mathchardef\nmv@citex@sfac1001\relax
  \sfcode`\.\nmv@citex@sfac
  \sfcode`\?\nmv@citex@sfac
  \sfcode`\!\nmv@citex@sfac
}
\ifnum\sfcode`\.=\@m
  \frenchspacing
\fi
%    \end{macrocode}
%\end{macro}
%\begin{macro}{\nmv@citex@nat}
%\begin{macro}{\nmv@activate}
%\begin{macro}{\cite}
% The swap has to be done at the beginning of the document. The
% internal flag from \pkg{natbib} is used, but under the
% circumstances we should be safe.  \cs{cite} is also patched to make
% the system active.
%    \begin{macrocode}
\newcommand*\nmv@citex@nat{}
\newcommand*\nmv@activate{%
  \let\nmv@citex@nat\@citex
  \let\@citex\nmv@citex
  \let\nmv@cite\cite
  \renewcommand*{\cite}[2][]{%
    \nmv@ifmtarg{##1}{%
      \nmv@citetrue
      \nmv@cite{##2}%
    }{%
      \nmv@citefalse
      \nmv@cite[##1]{##2}}%
  }%
}
%    \end{macrocode}
%\end{macro}
%\end{macro}
%\end{macro}
%\begin{macro}{\nmv@natbib@detect}
% So that \pkg{natbib} options can be set without worrying about load
% order, \pkg{natmove} doesn't require \pkg{natbib}. So a test is needed
% to see if it is actually loaded. This is done as a macro so that the
% effect can be changed by \pkg{achemso}.
%    \begin{macrocode}
\newcommand*\nmv@natbib@detect{%
  \@ifpackageloaded{natbib}{%
    \ifNAT@super
      \expandafter\nmv@activate
    \fi
  }{%
    \PackageInfo{natmove}{%
      The natbib package is not loaded.\MessageBreak
      Loading natmove will do nothing
    }%
  }%
}
\AtBeginDocument{\nmv@natbib@detect}
%    \end{macrocode}
%\end{macro}
%\begin{macro}{\natmovechars}
% A user macro is needed for moving characters.
%    \begin{macrocode}
\newcommand*\natmovechars{.,;:}
%</natmove>
%    \end{macrocode}
%\end{macro}
%
%\subsection{The configuration files}
%
% The configuration files for different journals are not very complex.
% Keeping everything separate simply helps with maintenance.
%
%    \begin{macrocode}
%<*achre4>
\ProvidesFile{achemso-achre4.cfg}
  [2019-02-14 v3.12a achemso configuration: Acc. Chem. Res.]
\setkeys{acs}{biblabel = plain}
\def\acs@type@list{article,suppinfo}
\def\abstractname{Conspectus}
%</achre4>
%<*aaembp>
\ProvidesFile{achemso-aaembp.cfg}
  [2019-02-14 v3.12a achemso configuration: ACS Appl. Electron. Mater.]
\def\acs@type@list{article,letter,suppinfo}
\setkeys{acs}{keywords = true}
%</aaembp>
%<*aaemcq>
\ProvidesFile{achemso-aaemcq.cfg}
  [2019-02-14 v3.12a achemso configuration: ACS Appl. Energy Mater.]
\def\acs@type@list{article,letter,suppinfo}
\setkeys{acs}{keywords = true}
%</aaemcq>
%<*aamick>
\ProvidesFile{achemso-aamick.cfg}
  [2019-02-14 v3.12a achemso configuration: ACS Appl. Mater. Interfaces]
\def\acs@type@list{article,letter,suppinfo}
\setkeys{acs}{keywords = true}
%</aamick>
%<*aanmf6>
\ProvidesFile{achemso-aanmf6.cfg}
  [2019-02-14 v3.12a achemso configuration: ACS Appl. Nano Mater.]
\def\acs@type@list{article,letter,suppinfo}
\setkeys{acs}{keywords = true}
%</aanmf6>
%<*aapmcd>
\ProvidesFile{achemso-aapmcd.cfg}
  [2019-02-14 v3.12a achemso configuration: ACS Appl. Polym. Mater.]
\def\acs@type@list{article,letter,suppinfo}
\setkeys{acs}{keywords = true}
%</aapmcd>
%<*abseba>
\ProvidesFile{achemso-abseba.cfg}
  [2019-02-14 v3.12a achemso configuration: ACS Biomater. Sci. Eng.]
\setkeys{acs}{keywords = true}
\def\acs@type@list{article,review,suppinfo}
%</abseba>
%<*accacs>
\ProvidesFile{achemso-accacs.cfg}
  [2019-02-14 v3.12a achemso configuration: ACS Catal.]
\setkeys{acs}{keywords = true}
\def\acs@type@list{article,letter,perspective,review,viewpoints,suppinfo}
\SectionNumbersOff
%</accacs>
%<*acscii>
\ProvidesFile{achemso-acscii.cfg}
  [2019-02-14 v3.12a achemso configuration: ACS Central Sci.]
\def\acs@type@list{article,review,suppinfo}
\setkeys{acs}{doi = true}
\SectionNumbersOff
%</acscii>
%<*acbcct>
\ProvidesFile{achemso-acbcct.cfg}
  [2019-02-14 v3.12a achemso configuration: ACS Chem. Biol.]
\setkeys{acs}{
  biblabel = fullstop,
  biochem  = true,
  super    = false
}
\def\acs@type@list{article,letter,review,suppinfo}
\SectionNumbersOff
%</acbcct>
%<*acncdm>
\ProvidesFile{achemso-acncdm.cfg}
  [2019-02-14 v3.12a achemso configuration: ACS Chem. Neurosci.]
\def\acs@type@list{article,review,letter,suppinfo,viewpoint}
\setkeys{acs}{
  biblabel = fullstop,
  biochem  = true,
  keywords = true,
  super    = false
}
%</acncdm>
%<*acsccc>
\ProvidesFile{achemso-acsccc.cfg}
  [2019-02-14 v3.12a achemso configuration: ACS Combinatorial Sci.]
\def\acs@type@list{article,letter,review,perspective,account,note,suppinfo}
\setkeys{acs}{keywords = true}
\SectionNumbersOff
%</acsccc>
%<*aesccq>
\ProvidesFile{achemso-aesccq.cfg}
  [2019-02-14 v3.12a achemso configuration: ACS Earth Space Chem.]
\def\acs@type@list{article,suppinfo}
\setkeys{acs}{keywords = true}
%</aesccq>
%<*aelccp>
\ProvidesFile{achemso-aelccp.cfg}
  [2019-02-14 v3.12a achemso configuration: ACS Energy Lett.]
\def\acs@type@list{letter,perspective,review,viewpoint,focus}
\SectionsOff
%</aelccp>
%<*ascefj>
\ProvidesFile{achemso-ascefj.cfg}
  [2019-02-14 v3.12a achemso configuration: ACS Infect. Dis.]
\def\acs@type@list{article,letter,perspective,review,viewpoint,suppinfo}
\setkeys{acs}{keywords = true}
\SectionsOff
%</ascefj>
%<*amlccd>
\ProvidesFile{achemso-amlccd.cfg}
  [2019-02-14 v3.12a achemso configuration: ACS Macro Lett.]
\def\acs@type@list{letter,suppinfo}
\SectionsOff
\setlength{\acs@tocentry@height}{8 cm}
\setlength{\acs@tocentry@width}{4 cm}
%</amlccd>
% \changes{v3.12a}{2019-02-14}{New \texttt{amlcef} configuration}
%<*amlcef>
\ProvidesFile{achemso-amlcef.cfg}
  [2019-02-14 v3.12a achemso configuration: ACS Mater. Lett.]
\def\acs@type@list{letter,perspective,review,viewpoint,focus}
\SectionsOff
%</amlcef>
%<*amclct>
\ProvidesFile{achemso-amclct.cfg}
  [2019-02-14 v3.12a achemso configuration: ACS Med. Chem. Lett.]
\def\acs@type@list{article,letter,perspective,suppinfo}
\SectionNumbersOff
%</amclct>
% \changes{v3.8n}{2015/01/13}{\emph{ACS Nano} want abbreviations}
% \changes{v3.11}{2018/04/02}{Change \emph{ACS Nano} truncation}
%<*ancac3>
\ProvidesFile{achemso-ancac3.cfg}
  [2019-02-14 v3.12a achemso configuration: ACS Nano]
\setkeys{acs}{
  abbreviations = true,
  biblabel      = fullstop,
  etalmode      = truncate,
  maxauthors    = 20,
  keywords      = true
}
\def\acs@type@list{article,perspective,suppinfo}
\newcommand*{\latin}[1]{\emph{#1}}
\SectionNumbersOff
%</ancac3>
%<*acsodf>
\ProvidesFile{achemso-acsodf.cfg}
  [2019-02-14 v3.12a achemso configuration: ACS Omega.]
\def\acs@type@list{article}
%</acsodf>
%<*apchd5>
\ProvidesFile{achemso-apchd5.cfg}
  [2019-02-14 v3.12a achemso configuration: ACS Photon.]
\def\acs@type@list{article,letter,perspective,review,suppinfo}
\setkeys{acs}{keywords = true}
\ifx\acs@manuscript\acs@manuscript@letter
  \SectionNumbersOff
\fi
%</apchd5>
%<*aidcbc>
\ProvidesFile{achemso-aidcbc.cfg}
  [2019-02-14 v3.12a achemso configuration: ACS Sensors]
\def\acs@type@list{article,letter,perspective,review,suppinfo}
\setkeys{acs}{keywords = true}
\SectionNumbersOff
%</aidcbc>
%<*ascecg>
\ProvidesFile{achemso-ascecg.cfg}
  [2019-02-14 v3.12a achemso configuration: ACS Sustainable Chem Eng.]
\setkeys{acs}{
  biblabel = fullstop,
  keywords = true
}
\SectionNumbersOff
\def\acs@type@list{article,feature,letter,perspective,review,suppinfo}
%</ascecg>
%<*asbcd6>
\ProvidesFile{achemso-asbcd6.cfg}
  [2019-02-14 v3.12a achemso configuration: ACS Synth. Biol.]
\setkeys{acs}{
  abbreviations = true,
  biblabel      = fullstop,
  biochem       = true,
  keywords      = true,
  super         = false
}
\def\acs@type@list{article,letter,note,tutorial,review,suppinfo}
%</asbcd6>
%<*ancham>
\ProvidesFile{achemso-ancham.cfg}
  [2019-02-14 v3.12a achemso configuration: Anal. Chem.]
\def\acs@type@list{article,note,suppinfo}
\SectionNumbersOff
%</ancham>
%<*bichaw>
\ProvidesFile{achemso-bichaw.cfg}
  [2019-02-14 v3.12a achemso configuration: Biochemistry]
\setkeys{acs}{
  abbreviations = true,
  biblabel      = brackets,
  biochem       = true,
}
\SectionNumbersOff
\def\acs@maketitle@extras@hook{%
  \par
  \acs@title@short@print
}
\g@addto@macro{\maketitle}{\newpage}
%    \end{macrocode}
%\begin{macro}{\acs@author@fnsymbol}
% Some changes to do with footnotes: symbols are different and symbol
% number one can only be used for the title.
%    \begin{macrocode}
\def\acs@author@fnsymbol#1{%
  \ensuremath{%
    \ifcase #1 *\or
      \dagger\or
        \ddagger\or
          \S\or
            \parallel\or
              \perp\or
                \P\or
                  \|\or
                    \bot\or
                      \#\or
                        @\or
                          \triangle\or
                            \nabla\else
                              #1%
    \fi
  }%
}
\def\@maketitle@title@hook{%
  \ifnum\acs@footnote@cnt>\@ne
  \else
    \global\acs@footnote@cnt\@ne
  \fi
}
%    \end{macrocode}
%\end{macro}
%    \begin{macrocode}
%</bichaw>
%<*bcches>
\ProvidesFile{achemso-bcches.cfg}
  [2019-02-14 v3.12a achemso configuration: Bioconjugate Chem.]
\setkeys{acs}{
  biochem = true,
  super   = false
}
\SectionNumbersOff
%</bcches>
%<*bomaf6>
\ProvidesFile{achemso-bomaf6.cfg}
  [2019-02-14 v3.12a achemso configuration: Biomacromolecules]
\def\acs@type@list{article,comment,communication,note,review,suppinfo}
\SectionNumbersOff
\AtEndOfClass{\SectionsOn}
%</bomaf6>
%<*crtoec>
\ProvidesFile{achemso-crtoec.cfg}
  [2019-02-14 v3.12a achemso configuration: Chem. Res. Toxicol.]
\setkeys{acs}{
  abbreviations = true,
  biochem       = true,
  keywords      = true
}
\def\acs@maketitle@extras@hook{%
  \par
  \acs@title@short@print
}
\def\acs@type@list{%
  article,perspective,profile,review,suppinfo%
}
\SectionNumbersOff
%</crtoec>
%<*chreay>
\ProvidesFile{achemso-chreay.cfg}
  [2019-02-14 v3.12a achemso configuration: Chem. Rev.]
\def\acs@type@default{review}
\def\acs@type@list{review}
%    \end{macrocode}
% The references section is numbered in \emph{Chem.\ Rev.}
%    \begin{macrocode}
\def\bibsection{\acs@section{\refname}}
%</chreay>
%<*cmatex>
%    \begin{macrocode}
\ProvidesFile{achemso-cmatex.cfg}
  [2019-02-14 v3.12a achemso configuration: Chem. Mater.]
\setkeys{acs}{keywords = true}
\SectionNumbersOff
\ifx\acs@manuscript\acs@manuscript@communication
  \AbstractOff
  \SectionsOff
\fi
%</cmatex>
%<*cgdefu>
\ProvidesFile{achemso-cgdefu.cfg}
  [2019-02-14 v3.12a achemso configuration: Cryst. Growth Des.]
\def\acs@type@list{%
  article,communication,perspective,suppinfo%
}
\SectionNumbersOff
\ifx\acs@manuscript\acs@manuscript@communication
  \SectionsOff
\fi
\setlength{\acs@tocentry@height}{8.9 cm}
\setlength{\acs@tocentry@width}{4.6 cm}
%</cgdefu>
%<*enfuem>
\ProvidesFile{achemso-enfuem.cfg}
  [2019-02-14 v3.12a achemso configuration: Energy Fuels]
\def\acs@type@list{article,review,suppinfo}
\SectionNumbersOff
%</enfuem>
%<*esthag>
\ProvidesFile{achemso-esthag.cfg}
  [2019-02-14 v3.12a achemso configuration: Environ. Sci. Technol.]
\def\acs@type@list{article,suppinfo}
\SectionNumbersOff
%</esthag>
%<*estlcu>
\ProvidesFile{achemso-estlcu.cfg}
  [2019-02-14 v3.12a achemso configuration: Environ. Sci. Technol. Lett.]
\def\acs@type@list{letter,suppinfo}
\SectionNumbersOff
%</estlcu>
% \changes{v3.8m}{2014/11/24}{Fixes for \emph{Ind.\ Eng.\ Chem.\ Res.}\ style}
%<*iecred>
\ProvidesFile{achemso-iecred.cfg}
  [2019-02-14 v3.12a achemso configuration: Ind. Eng. Chem. Res.]
\setkeys{acs}{biblabel = brackets}
\SectionNumbersOff
%</iecred>
%<*inoraj>
\ProvidesFile{achemso-inoraj.cfg}
  [2019-02-14 v3.12a achemso configuration: Inorg. Chem.]
\SectionNumbersOff
\ifx\acs@manuscript\acs@manuscript@communication
  \AbstractOff
  \SectionsOff
\fi
\setkeys{acs}{doi = true}
%</inoraj>
%<*jafcau>
\ProvidesFile{achemso-jafcau.cfg}
  [2019-02-14 v3.12a achemso configuration: J. Agric. Food Chem.]
\setkeys{acs}{keywords = true}
\def\acs@type@list{article,suppinfo}
\SectionNumbersOff
%</jafcau>
%<*jceda8>
\ProvidesFile{achemso-jceda8.cfg}
  [2019-02-14 v3.12a achemso configuration: J. Chem. Ed.]
\setkeys{acs}{keywords = true}
\SectionNumbersOff
\def\acs@type@list{article,suppinfo}
\SectionNumbersOff
%</jceda8>
%<*jceaax>
\ProvidesFile{achemso-jceaax.cfg}
  [2019-02-14 v3.12a achemso configuration: J. Chem. Eng. Data]
\def\acs@type@list{article,suppinfo}
\def\refname{Literature Cited}
\SectionNumbersOff
%</jceaax>
%<*jcisd8>
\ProvidesFile{achemso-jcisd8.cfg}
  [2019-02-14 v3.12a achemso configuration: J. Chem. Inf. Model.]
\def\acs@type@list{article,suppinfo}
\SectionNumbersOff
%</jcisd8>
%<*jctcce>
\ProvidesFile{achemso-jctcce.cfg}
  [2019-02-14 v3.12a achemso configuration: J. Chem. Theory Comput.]
\def\acs@type@list{article,suppinfo}
\setkeys{acs}{maxauthors = 0}
%</jctcce>
%<*jmcmar>
\ProvidesFile{achemso-jmcmar.cfg}
  [2019-02-14 v3.12a achemso configuration: J. Med. Chem.]
\def\acs@type@list{article,letter,perspective,suppinfo}
\SectionNumbersOff
\setlength{\acs@tocentry@height}{5.5cm}
\setlength{\acs@tocentry@width}{21cm}
%</jmcmar>
%<*jnprdf>
\ProvidesFile{achemso-jnprdf.cfg}
  [2019-02-14 v3.12a achemso configuration: J. Nat. Prod.]
\renewcommand*{\abstractname}{ABSTRACT}
\renewcommand*{\acknowledgementname}{ACKNOWLEDGEMENT}
\renewcommand*{\suppinfoname}{ASSOCIATED CONTENT}
\SectionNumbersOff
\renewcommand*{\acs@abstract}{%
  \quotation
  \textbf{\abstractname :}%
}
\AbstractOn
\ifx\acs@manuscript\acs@manuscript@communication
  \AbstractOff
  \SectionsOff
\fi
\def\refname{REFERENCES}
\captionsetup{
  labelfont = bf,
  labelsep  = period
}
%</jnprdf>
%<*joceah>
\ProvidesFile{achemso-joceah.cfg}
  [2019-02-14 v3.12a achemso configuration: J. Org. Chem.]
\ifx\acs@manuscript\acs@manuscript@communication
  \AbstractOff
  \SectionsOff
\fi
\floatstyle{plaintop}
\restylefloat{scheme}
\floatstyle{plain}
\renewcommand*\acs@type@list{article,communication,note,perspective,review,suppinfo}
%</joceah>
%    \end{macrocode}
% \changes{v3.8n}{2015/01/13}{Drop use of keywords for \emph{J.~Phys.\ Chem.}\
%   A, B and C}
%    \begin{macrocode}
%<*jpcafh>
\ProvidesFile{achemso-jpcafh.cfg}
  [2019-02-14 v3.12a achemso configuration: J. Phys. Chem. A]
\def\acs@type@list{article,suppinfo}
\setkeys{acs}{
  etalmode     = truncate,
  maxauthors   = 10
}
\SectionNumbersOff
\captionsetup[table]{labelfont=bf,textfont=bf}
%    \end{macrocode}
% Title
%    \begin{macrocode}
\g@addto@macro{\maketitle}{\newpage}
%</jpcafh>
%<*jpcbfk>
\ProvidesFile{achemso-jpcbfk.cfg}
  [2019-02-14 v3.12a achemso configuration: J. Phys. Chem. B]
\def\acs@type@list{article,suppinfo}
\setkeys{acs}{
  etalmode     = truncate,
  maxauthors   = 10
}
\SectionNumbersOff
\captionsetup[table]{labelfont=bf,textfont=bf}
\g@addto@macro{\maketitle}{\newpage}
%</jpcbfk>
%<*jpccck>
\ProvidesFile{achemso-jpccck.cfg}
  [2019-02-14 v3.12a achemso configuration: J. Phys. Chem. C]
\def\acs@type@list{article,suppinfo}
\setkeys{acs}{
  etalmode     = truncate,
  maxauthors   = 10
}
\SectionNumbersOff
\captionsetup[table]{labelfont=bf,textfont=bf}
\g@addto@macro{\maketitle}{\newpage}
%</jpccck>
%    \end{macrocode}
% \changes{v3.8m}{2014/08/24}{\emph{J.~Phys.\ Chem.\ Lett.}\ uses a
% non-standard size for the TOC graphic}
%    \begin{macrocode}
%<*jpclcd>
\ProvidesFile{achemso-jpclcd.cfg}
  [2019-02-14 v3.12a achemso configuration: J. Phys. Chem. Lett.]
\setkeys{acs}{
  etalmode   = truncate,
  maxauthors = 10
}
\def\acs@type@default{letter}
\def\acs@type@list{letter}
\SectionNumbersOff
\captionsetup[table]{labelfont=bf,textfont=bf}
\g@addto@macro{\maketitle}{\newpage}
\renewcommand{\acs@tocentry@print}[1]{%
  \gdef\acs@tocentry@text{#1}%
}
\renewcommand*{\acs@abstract@extras}{%
  \begingroup
    \acs@tocentry@print@aux
  \endgroup
  \acs@keywords@print
  \newpage
}
\setlength\acs@tocentry@height{2in}
\setlength\acs@tocentry@width{2in}
%</jpclcd>
%<*jprobs>
\ProvidesFile{achemso-jprobs.cfg}
  [2019-02-14 v3.12a achemso configuration: J. Proteome Res.]
\setkeys{acs}{keywords = true}
\def\acs@type@list{article,review,suppinfo}
\SectionNumbersOff
%</jprobs>
%    \end{macrocode}
% \changes{v3.12a}{2019-02-14}{\emph{J.~Am.\ Chem.\ Soc.}\ now requires titles}
% The \emph{J.~Am.\ Chem.\ Soc.}~configuration is rather more
% complicated as there is the need to construct a ``galley-like''
% layout for communications.
%    \begin{macrocode}
%<*jacsat>
\ProvidesFile{achemso-jacsat.cfg}
  [2019-02-14 v3.12a achemso configuration: J. Am. Chem. Soc.]
\SectionNumbersOff
\ifx\acs@manuscript\acs@manuscript@communication\else
  \expandafter\endinput
\fi
\setkeys{acs}{articletitle = true}
%    \end{macrocode}
% Everything from here onward applies only to communications. Some
% adjustments are now made using the existing tools.
%    \begin{macrocode}
\setkeys{acs}{
  email  = true,
  layout = twocolumn
}
\SectionsOff
\acs@layout@nine
%    \end{macrocode}
%      
%\begin{environment}{abstract}
% The abstract appears at the start of the document, with lines around
% it.s
%    \begin{macrocode}
\renewenvironment{abstract}{%
  \hrule
  \vspace{2 mm}%
  \sffamily
  \noindent
  \emph{\textbf{Abstract:}}%
}{%
  \vspace{2 mm}%
  \hrule
  \vspace{6 mm}%
}
%    \end{macrocode}
%\end{environment}
%
%\begin{macro}{\acksize}
%\begin{macro}{\affilsize}
%\begin{macro}{\authorsize}
%\begin{macro}{\emailsize}
%\begin{macro}{\capsize}
%\begin{macro}{\refsize}
%\begin{macro}{\suppsize}
%\begin{macro}{\titlesize}
% To keep things logical, the size macros are given names related to
% their function. Some of these are new, some are defined in the class.
%    \begin{macrocode}
\def\affilsize{%
  \@setfontsize\affilsize\@ixpt\@xpt
}
\def\acksize{%
  \@setfontsize\acksize\@ixpt\@xipt
}
\def\authorsize{%
  \@setfontsize\authorsize{10.5}{12.5}%
}
\newcommand*\capsize{%
  \@setfontsize\capsize\@viiipt\@ixpt
}
\def\emailsize{%
  \@setfontsize\emailsize\@viiipt{15}%
}
\newcommand*\refsize{%
  \@setfontsize\refsize{7.5}{7.5}%
}
\def\suppsize{%
  \@setfontsize\suppsize{8.5}{10.5}%
}
\def\titlesize{%
  \@setfontsize\titlesize\@xiipt{13}%
}
\let\footnotesize\refsize
\let\captionfont\capsize
%    \end{macrocode}
%\end{macro}
%\end{macro}
%\end{macro}
%\end{macro}
%\end{macro}
%\end{macro}
%\end{macro}
%\end{macro}
%\begin{macro}{\emailfont}
% A slight font change.
%    \begin{macrocode}
\def\emailfont{\sffamily}
%    \end{macrocode}
%\end{macro}
%\begin{macro}{\ps@plain}
%\begin{macro}{\ps@jacs}
% The header styles are done the hard way, to keep down the number of
% packages loaded.
%    \begin{macrocode}
\def\ps@plain{%
  \let\@mkboth\@gobbletwo
  \let\@oddhead\@empty
  \def\@oddfoot{%
    \reset@font
    \sffamily
    \textbf{\thepage}%
    \hfil
  }%
  \let\@evenhead\@empty
  \let\@evenfoot\@oddfoot
}
\def\ps@acs{%
  \def\@oddfoot{%
    \reset@font
    \sffamily
    \textbf{\thepage}%
    \hfil
  }%
  \def\@evenfoot{%
    \reset@font
    \hfil
    \sffamily
    \textbf{\thepage}%
  }%
  \def\@oddhead{}%
  \let\@evenhead\@oddhead
}
\pagestyle{acs}
%    \end{macrocode}
%\end{macro}
%\end{macro}
%\begin{macro}{\acs@space@pre@title}
%\begin{macro}{\acs@space@post@title}
%\begin{macro}{\acs@space@post@author}
%\begin{macro}{\acs@space@post@address}
%\begin{macro}{\acs@space@post@email}
% Length adjustments for the title.
%    \begin{macrocode}
\setlength\acs@space@pre@title{16mm}
\setlength\acs@space@post@title{0mm}
\setlength\acs@space@post@author{0mm}
\setlength\acs@space@post@address{0mm}
\setlength\acs@space@post@email{-1mm}
\setlength\acs@maketitle@width{152.4mm}
%    \end{macrocode}
%\end{macro}
%\end{macro}
%\end{macro}
%\end{macro}
%\end{macro}
%\begin{macro}{\acs@contact@details}
% Contact details are different here.
%    \begin{macrocode}
\def\acs@contact@details{%
  Received \today; E-mail: \acs@email@list
}
%    \end{macrocode}
%\end{macro}
%\begin{macro}{\acs@maketitle@extras}
% No keywords or abbreviations for \emph{J.~Am.\ Chem.\ Soc.}.
%    \begin{macrocode}
\let\acs@maketitle@extras\relax
%    \end{macrocode}
%\end{macro}
% Getting the floats correct is a difficult task ``by hand''; using
% the \pkg{caption} package makes this a lot easier.
%    \begin{macrocode}
\floatstyle{plaintop}
\restylefloat{scheme}
\floatstyle{plain}
\DeclareCaptionLabelSeparator{perquad}{.\quad}
\captionsetup{
  singlelinecheck = off,
  labelfont       = {bf,it,sf},
  textfont        = sf,
  labelsep        = perquad
}
\captionsetup[figure]{textfont=rm}
%    \end{macrocode}
%\begin{macro}{\acs@table}
%\begin{macro}{\table}
% A patch or \cs{table}.
%    \begin{macrocode}
\newcommand*\acs@table{}
\let\acs@table\table
\def\table{%
  \capsize
  \acs@table
}
%    \end{macrocode}
%\end{macro}
%\end{macro}
% The bibliography has to be adjusted.
%    \begin{macrocode}
\AtBeginDocument{%
  \def\bibsection{%
    \@startsection
      {section}
      {1}
      {\z@}{\z@}{2.5mm}%
      {\normalfont\acksize\bfseries}
      {\hrule\nobreak\vspace{1.2mm}\noindent\refname}%
  }%
  \let\bibfont\refsize
  \setlength{\bibhang}{0.61cm}%
  \setlength{\bibsep}{0mm}%
}
%</jacsat>
%<*langd5>
\ProvidesFile{achemso-langd5.cfg}
  [2019-02-14 v3.12a achemso configuration: Langmuir]
\def\acs@type@list{article,letter,suppinfo}
\SectionNumbersOff
%</langd5>
%<*mamobx>
\ProvidesFile{achemso-mamobx.cfg}
  [2019-02-14 v3.12a achemso configuration: Macromolecules]
\SectionNumbersOff
\setkeys{acs}{doi = true}
%</mamobx>
%<*mpohbp>
\ProvidesFile{achemso-mpohbp.cfg}
  [2019-02-14 v3.12a achemso configuration: Mol. Pharm.]
\setkeys{acs}{keywords = true}
\def\acs@type@list{article,suppinfo}
\SectionNumbersOff
%</mpohbp>
%<*nalefd>
\ProvidesFile{achemso-nalefd.cfg}
  [2019-02-14 v3.12a achemso configuration: Nano Lett.]
\setkeys{acs}{keywords = true}
\def\acs@type@default{letter}
\def\acs@type@list{letter}
\SectionNumbersOff
%</nalefd>
%<*orlef7>
\ProvidesFile{achemso-orlef7.cfg}
  [2019-02-14 v3.12a achemso configuration: Org. Lett.]
\def\acs@type@default{communication}
\def\acs@type@list{communication}
\SectionNumbersOff
\setkeys{acs}{layout = twocolumn}
\RequirePackage{xcolor}
\definecolor{orglett}{RGB}{128,0,0}
%</orlef7>
%<*oprdfk>
\ProvidesFile{achemso-oprdfk.cfg}
  [2019-02-14 v3.12a achemso configuration: Org. Proc. Res. Dev.]
\def\acs@type@list{article,highlight,review,suppinfo}
\SectionNumbersOff
%</oprdfk>
%<*orgnd7>
\ProvidesFile{achemso-orgnd7.cfg}
  [2019-02-14 v3.12a achemso configuration: Organometallics]
\SectionNumbersOff
%</orgnd7>
%    \end{macrocode}
%
%\Finale
%
%\iffalse
%<*bst>
% Database entries

ENTRY
  {
    address
    author
    booktitle
    chapter
    ctrl-article-title
    ctrl-chapter-title
    ctrl-doi
    ctrl-etal-firstonly
    ctrl-etal-number
    ctrl-use-title
    doi
    edition
    editor
    howpublished
    institution
    journal
    key
    maintitle
    note
    number
    organization
    pages
    publisher
    school
    series
    title
    type
    url
    version
    volume
    year
  }
  { }
  {
    label
    short.names
  }
 
% Generic logic functions, from the core BibTeX styles

FUNCTION { and } {
    { }
    {
      pop$
      #0
    }
  if$ 
}

FUNCTION { not } {
    { #0 }
    { #1 }
  if$ 
}

FUNCTION { or } {
    {
      pop$
      #1
    }
    { }
  if$ 
}

FUNCTION { xor } {
    { not }
    { }
  if$ 
}
 
% Generic functions for basic manipulations: many of these
% come from the core BibTeX styles or 'Tame the BeaST'

FUNCTION { chr.to.value.error } {
  #48 +
  int.to.chr$
  "'" swap$ *
  "'" *
  " is not a number: treated as zero." *
  warning$
  #0
}

FUNCTION { chr.to.value } {
  chr.to.int$ #48 -
  duplicate$
  #0 <
    { chr.to.value.error }
    {
      duplicate$
      #9 >
        { chr.to.value.error }
        { }
      if$ 
    }
  if$ 
}

STRINGS {
  extract.input.str
  extract.output.str
}

FUNCTION { is.a.digit } {
  duplicate$
  "" =
    {
      pop$
      #0
    }
    {
      chr.to.int$
      #48 -
      duplicate$
     #0 < swap$ #9 > or not
   }
  if$
}

FUNCTION{ is.a.number } {
  {
    duplicate$
    #1 #1 substring$
    is.a.digit
  }
    { #2 global.max$ substring$ }
  while$
  "" =
}

FUNCTION { extract.number } {
  duplicate$
  'extract.input.str :=
  "" 'extract.output.str :=
  { extract.input.str empty$ not }
    {
      extract.input.str #1 #1 substring$
      extract.input.str #2 global.max$ substring$ 'extract.input.str :=
      duplicate$
      is.a.number
        { extract.output.str swap$ * 'extract.output.str := }
        {
          pop$
          "" 'extract.input.str :=
        }
      if$
    }
  while$
  extract.output.str empty$
    { }
    {
      pop$
      extract.output.str
    }
  if$
}

FUNCTION { field.or.null } {
  duplicate$
  empty$
    {
      pop$
      ""
    }
    { }
  if$
}

INTEGERS {
  multiply.a.int
  multiply.b.int
}

FUNCTION { multiply } {
  'multiply.a.int :=
  'multiply.b.int :=
  multiply.b.int #0 <
    {
      #-1
      #0 multiply.b.int - 'multiply.b.int :=
    }
    { #1 }
  if$
  #0
  { multiply.b.int #0 > } 
    {
      multiply.a.int +
      multiply.b.int #1 - 'multiply.b.int :=
    }
  while$
  swap$
    { }
    { #0 swap$ - }
  if$ 
}

INTEGERS { str.conversion.int }

FUNCTION { str.to.int.aux.ii } {
  {
    duplicate$
    empty$ not
  }
    {
      swap$
      #10 multiply 'str.conversion.int :=
      duplicate$
      #1 #1 substring$
      chr.to.value
      str.conversion.int +
      swap$
      #2 global.max$ substring$
    }
  while$
  pop$
}

FUNCTION { str.to.int.aux.i } {
  duplicate$
  #1 #1 substring$
  "-" =
   {
      #1 swap$
      #2 global.max$ substring$
    }
    {
      #0 swap$
    }
  if$ 
  #0
  swap$
  str.to.int.aux.ii
  swap$
    { #0 swap$ - }
    { }
  if$
}

FUNCTION { str.to.int } {
  duplicate$
  empty$
    {
      pop$
      #0
    }
    { str.to.int.aux.i }
  if$ 
}

FUNCTION { tie.or.space.connect } {
 duplicate$
 text.length$ #3 <
    { "~" }
    { " " }
  if$
  swap$ * *
}

FUNCTION { yes.no.to.bool } {
  duplicate$
  empty$
    {
      pop$
      #0
    }
    {
      "l" change.case$
      "yes" =
        { #1 }
        { #0 }
      if$ 
    }
  if$ 
}

% Functions of formatting

FUNCTION { bold } {
  duplicate$
  empty$
    {
      pop$
      ""
    }
    { "\textbf{" swap$ * "}" * }
  if$
}

FUNCTION { emph } {
  duplicate$
  empty$
    {
      pop$
      ""
    }
    { "\emph{" swap$ * "}" * }
  if$
}


FUNCTION { paren } {
  duplicate$
  empty$
    {
      pop$
      ""
    }
    { "(" swap$ * ")" * }
  if$
}

% Functions for punctuation

FUNCTION { add.comma } { ", " * }

FUNCTION { add.colon } {  ": " * }

FUNCTION { add.period } { add.period$ " " * }

FUNCTION { add.semicolon } { "; " * }

FUNCTION { add.space } { " " * }

% Bibliography strings: fixed values collected into functions

FUNCTION { bbl.and }     { "and" }

FUNCTION { bbl.chapter } { "Chapter" }

FUNCTION { bbl.doi }     { "DOI:" }

FUNCTION { bbl.editor }  { "Ed." }

FUNCTION { bbl.editors } { "Eds." }

FUNCTION { bbl.edition } { "ed." }

FUNCTION { bbl.etal }    { "\latin{et~al.}" }

FUNCTION { bbl.in }      { "In" }

FUNCTION { bbl.inpress } { "in press" }

FUNCTION { bbl.msc }     { "M.Sc.\ thesis" }

FUNCTION { bbl.page }    { "p" }

FUNCTION { bbl.pages }   { "pp" }

FUNCTION { bbl.phd }     { "Ph.D.\ thesis" }

FUNCTION { bbl.version } { "version" }

FUNCTION { bbl.volume }  { "Vol." }

% Functions for number formatting

STRINGS { pages.str }

FUNCTION { hyphen.to.dash } {
  'pages.str :=
  ""
  { pages.str empty$ not }
    {
      pages.str #1 #1 substring$
      "-" =
        {
          "--" *
          {
            pages.str #1 #1 substring$
            "-" =
          }
            { pages.str #2 global.max$ substring$ 'pages.str := }
          while$
        }
        {
          pages.str #1 #1 substring$
          *
          pages.str #2 global.max$ substring$ 'pages.str :=
        }
      if$
    }
  while$
}

INTEGERS { multiresult.bool }

FUNCTION { multi.page.check } {
  'pages.str :=
  #0 'multiresult.bool :=
  {
    multiresult.bool not
    pages.str empty$ not
    and
  }
    {
      pages.str #1 #1 substring$
      duplicate$
      "-" = swap$ duplicate$
      "," = swap$
      "+" =
      or or
        { #1 'multiresult.bool := }
        { pages.str #2 global.max$ substring$ 'pages.str := }
      if$
    }
  while$
  multiresult.bool
}

% Functions for calculating the label data needed by natbib


INTEGERS {
  current.name.int
  names.separate.comma
  names.separate.semicolon
  names.separate.comma.bool
  remaining.names.int
  total.names.int
}

STRINGS {
  current.name.str
  names.str
}

FUNCTION { full.format.names } {
  'names.str :=
  #1 'current.name.int :=
  names.str num.names$ 'remaining.names.int :=
  { remaining.names.int #0 > }
    {
      names.str current.name.int "{vv~}{ll}" format.name$
      current.name.int #1 >
        {
          swap$ add.comma swap$
          remaining.names.int #1 >
            { }
            {
              duplicate$
              "others" =
                { bbl.etal }
                { bbl.and }
              if$
              add.space swap$ *
            }
          if$
          *
        }
        { }
      if$   
      remaining.names.int #1 - 'remaining.names.int :=
      current.name.int #1 + 'current.name.int :=
    }
  while$
}

FUNCTION { full.author } {
  author empty$
    { "" }
    { author full.format.names }
  if$ 
}

FUNCTION { full.author.editor } {
  author empty$
    {
      editor empty$
        { "" }
        { editor full.format.names }
      if$
    }
    { author full.format.names }
  if$ 
}

FUNCTION { full.editor } {
  editor empty$
    { "" }
    { editor full.format.names }
  if$ 
}

FUNCTION { short.format.names } {
  'names.str :=
  names.str #1 "{vv~}{ll}" format.name$
  names.str num.names$
  duplicate$
  #2 >
    {
      pop$
      add.space bbl.etal *
    }
    {
      #2 <
        { }
        {
          names.str #2 "{ff }{vv }{ll}{ jj}" format.name$
          "others" =
            { add.space bbl.etal * }
            {
              add.space
              bbl.and add.space
              *
              names.str #2 "{vv~}{ll}" format.name$
              *
            }
          if$
        }
      if$ 
    }
  if$ 
}

FUNCTION { short.author.key } {
  author empty$
    {
      key empty$
        { cite$ #1 #3 substring$ }
        { key }
      if$ 
    }
    { author short.format.names }
  if$ 
}

FUNCTION { short.author.editor.key } {
  author empty$
    {
      editor empty$
        {
          key empty$
            { cite$ #1 #3 substring$ }
            { key }
          if$ 
        }
        { editor short.format.names }
      if$
    }
    { author short.format.names }
  if$
}

FUNCTION { short.author.key.organization } {
  author empty$
    {
      key empty$
        {
          organization empty$
            { cite$ #1 #3 substring$ }
            {
              organization #1 #4 substring$
              "The " =
                { organization }
                { organization #5 global.max$ substring$ }
              if$
              #3 text.prefix$
            }
          if$ 
        }
        { key }
      if$ 
    }
    { author short.format.names }
  if$ 
}

FUNCTION { short.editor.key.organization } {
  editor empty$
    {
      key empty$
        {
          organization empty$
            { cite$ #1 #3 substring$ }
            {
              organization #1 #4 substring$
              "The " =
                { organization }
                { organization #5 global.max$ substring$ }
              if$
              #3 text.prefix$
            }
          if$ 
        }
        { key }
      if$ 
    }
    { editor short.format.names }
  if$ 
}

FUNCTION { calculate.full.names } {
  type$ "book" =
  type$ "inbook" =
  or
    { full.author.editor }
    {
      type$ "proceedings" =
        { full.editor }
        { full.author }
      if$ 
    }
  if$
}

FUNCTION { calculate.short.names } {
  type$ "book" =
  type$ "inbook" =
  or
    { short.author.editor.key }
    {
      type$ "proceedings" =
        { short.editor.key.organization }
        {
          type$ "manual" =
            { short.author.key.organization }
            { short.author.key }
          if$ 
        }
      if$ 
    }
  if$
  'short.names :=
}

FUNCTION { calculate.names } {
  calculate.short.names
  short.names
  year empty$
    { "()" }
    { "(" year * ")" * }
  if$
  *
  'label :=
}

% Counting up the number of entries

INTEGERS { entries.int }

FUNCTION { initialize.count.entries } {
  #0 'entries.int  :=
}

FUNCTION { count.entries } {
  entries.int #1 + 'entries.int  :=
}

% Start and end of bibliography functions

FUNCTION { begin.bib } {
  "achemso 2019-02-14 v3.12a" top$
  preamble$ empty$
    { }
    {
      preamble$ write$
      newline$
    }
  if$
  "\providecommand{\latin}[1]{#1}" write$
  newline$
  "\makeatletter" write$
  newline$
  "\providecommand{\doi}" write$
  newline$
  "  {\begingroup\let\do\@makeother\dospecials" write$
  newline$
  "  \catcode`\{=1 \catcode`\}=2 \doi@aux}" write$
  newline$
  "\providecommand{\doi@aux}[1]{\endgroup\texttt{#1}}" write$
   newline$
  "\makeatother" write$
  newline$
  "\providecommand*\mcitethebibliography{\thebibliography}" write$
  newline$
  "\csname @ifundefined\endcsname{endmcitethebibliography}" write$
  "  {\let\endmcitethebibliography\endthebibliography}{}" write$
  newline$
  "\begin{mcitethebibliography}{"
    entries.int int.to.str$  * "}" * write$
  newline$
  "\providecommand*\natexlab[1]{#1}" write$
  newline$
  "\providecommand*\mciteSetBstSublistMode[1]{}" write$
  newline$
  "\providecommand*\mciteSetBstMaxWidthForm[2]{}" write$
  newline$
  "\providecommand*\mciteBstWouldAddEndPuncttrue" write$
  newline$
  "  {\def\EndOfBibitem{\unskip.}}" write$
  newline$
  "\providecommand*\mciteBstWouldAddEndPunctfalse" write$
  newline$
  "  {\let\EndOfBibitem\relax}" write$
  newline$
  "\providecommand*\mciteSetBstMidEndSepPunct[3]{}" write$
  newline$
  "\providecommand*\mciteSetBstSublistLabelBeginEnd[3]{}" write$
  newline$
  "\providecommand*\EndOfBibitem{}" write$
  newline$
  "\mciteSetBstSublistMode{f}" write$
  newline$
  "\mciteSetBstMaxWidthForm{subitem}{(\alph{mcitesubitemcount})}" write$
  newline$
  "\mciteSetBstSublistLabelBeginEnd" write$
  newline$
  "  {\mcitemaxwidthsubitemform\space}" write$
  newline$
  "  {\relax}" write$
  newline$
  "  {\relax}" write$
  newline$    
}

FUNCTION { end.bib } {
  newline$
  "\end{mcitethebibliography}" write$
  newline$
}

% Functions used for the special "control" entry type, to pass data
% from LaTeX to BibTeX

INTEGERS {
  ctrl.article.title.bool
  ctrl.chapter.title.bool
  ctrl.doi.bool
  ctrl.etal.firstonly.bool
  ctrl.etal.number.int
}

FUNCTION { initialize.control.values } {
  #1 'ctrl.article.title.bool :=
  #0 'ctrl.chapter.title.bool :=
  #0 'ctrl.doi.bool :=
  #1 'ctrl.etal.firstonly.bool :=
  #15 'ctrl.etal.number.int :=
}

FUNCTION { control } {
  ctrl-article-title  yes.no.to.bool 'ctrl.article.title.bool  :=
  ctrl-chapter-title  yes.no.to.bool 'ctrl.chapter.title.bool  :=
  ctrl-doi            yes.no.to.bool 'ctrl.doi.bool            :=
  ctrl-etal-firstonly yes.no.to.bool 'ctrl.etal.firstonly.bool :=
  ctrl-etal-number    str.to.int     'ctrl.etal.number.int     :=
  ctrl-use-title empty$
    'skip$
    { ctrl-use-title yes.no.to.bool 'ctrl.article.title.bool  := }
  if$
}

% Functions of each bibitem: tracking punctuation and transferring
% items to the .bbl file

INTEGERS {
  next.punct.comma
  next.punct.period
  next.punct.semicolon
  next.punct.space
}

FUNCTION { initialize.tracker } {
  #0 'next.punct.comma     :=
  #1 'next.punct.period    :=
  #2 'next.punct.semicolon :=
  #3 'next.punct.space     :=
}

INTEGERS { next.punct.int }

FUNCTION { output } {
  swap$
  duplicate$ empty$
    { pop$ }
    {
      next.punct.int next.punct.space =
        { add.space }
        {
          next.punct.int next.punct.comma =
            { add.comma }
            {
              next.punct.int next.punct.semicolon =
                { add.semicolon }
                { add.period }
              if$ 
            }
          if$ 
        }
      if$
      write$
    }
  if$  
  next.punct.comma 'next.punct.int :=
}

% Functions for each bibliography entry: start and finish

FUNCTION { begin.bibitem } {
  newline$
  "\bibitem" write$
  label
  calculate.full.names
  duplicate$
  short.names =
    { pop$ }
    { * }
  if$ 
  "[" swap$ * "]" * write$
  "{" cite$ * "}" * write$
  newline$
  ""
  next.punct.comma 'next.punct.int :=
}

INTEGERS { add.period.length.int }

FUNCTION { would.add.period } {
  duplicate$
  add.period$
  text.length$ 'add.period.length.int :=
  duplicate$
  text.length$
  add.period.length.int =
    { #0 }
    { #1 }
  if$
}

FUNCTION { end.bibitem } {
  would.add.period
    {
      "\relax" * write$
      newline$
      "\mciteBstWouldAddEndPuncttrue" write$
      newline$
      "\mciteSetBstMidEndSepPunct{\mcitedefaultmidpunct}" write$
      newline$
      "{\mcitedefaultendpunct}{\mcitedefaultseppunct}\relax"
    }
    {
      "\relax" * write$
      newline$
      "\mciteBstWouldAddEndPunctfalse" write$
       newline$
      "\mciteSetBstMidEndSepPunct{\mcitedefaultmidpunct}" write$
      newline$
      "{}{\mcitedefaultseppunct}\relax"
    }
  if$
  write$
  newline$
  "\EndOfBibitem" write$
}

% Formatting names: authors and editors are not quite the same,
% and there is the question of how to handle 'et al.'

FUNCTION { initialize.name.separator } {
  #1 'names.separate.comma     :=
  #0 'names.separate.semicolon :=
}

FUNCTION { format.names.loop } {
  { remaining.names.int #0 > }
    {
      names.str current.name.int "{vv~}{ll,}{~f.}{,~jj}" format.name$
      duplicate$
      'current.name.str :=
      current.name.int #1 >
        {
          duplicate$
          "others," =
            {
              pop$
              *
              bbl.etal
              add.space
              remaining.names.int #1 - 'remaining.names.int :=
            }
            { 
              swap$
              names.separate.comma.bool
                { add.comma }
                { add.semicolon } 
              if$
%<*bio>
              remaining.names.int #1 >
                { }
                { bbl.and add.space * }
              if$ 
%</bio>s
              swap$
              *
            }
          if$  
        }
        { }
      if$   
      remaining.names.int #1 - 'remaining.names.int :=
      current.name.int #1 + 'current.name.int :=
    }
  while$
}

FUNCTION { format.names.all } {
  total.names.int 'remaining.names.int :=
  format.names.loop
}

FUNCTION { format.names.etal } {
  ctrl.etal.firstonly.bool
    { #1 'remaining.names.int := }
    { ctrl.etal.number.int 'remaining.names.int := }
  if$ 
  format.names.loop
  current.name.str "others," =
    { }
    {
      add.space
      bbl.etal
      add.space
      *
    }
  if$ 
}

FUNCTION { format.names } {
  'names.separate.comma.bool :=
  'names.str :=
  #1 'current.name.int :=
  names.str num.names$ 'total.names.int :=
  total.names.int ctrl.etal.number.int >
    {
      ctrl.etal.number.int #0 =
        { format.names.all }
        { format.names.etal }
      if$ 
    }
    { format.names.all }
  if$ 
}

% Converting editions into their fixed representations

FUNCTION { bbl.first }  { "1st" }

FUNCTION { bbl.second } { "2nd" }

FUNCTION { bbl.third }  { "3rd" }

FUNCTION { bbl.fourth } { "4th" }

FUNCTION { bbl.fifth }  { "5th" }

FUNCTION { bbl.st }     { "st" }

FUNCTION { bbl.nd }     { "nd" }

FUNCTION { bbl.rd }     { "rd" }

FUNCTION { bbl.th }     { "th" }

STRINGS {
  ord.input.str
  ord.output.str
}

FUNCTION { make.ordinal } {
  duplicate$
  "1" swap$
  *
  #-2 #1 substring$
  "1" =
    {
      bbl.th *
    }
    {
      duplicate$
      #-1 #1 substring$
      duplicate$
      "1" =
        {
          pop$
          bbl.st *
        }
        {
          duplicate$
          "2" =
            {
              pop$
              bbl.nd *
            }
            {
              "3" =
                { bbl.rd * }
                { bbl.th * }
              if$ 
            }
          if$
        }
      if$
    }
  if$ 
}

FUNCTION { convert.to.ordinal } {
  extract.number
  "l" change.case$ 'ord.input.str :=
  ord.input.str "first" = ord.input.str "1" = or
    { bbl.first 'ord.output.str := }
    {
      ord.input.str "second" = ord.input.str "2" = or
        { bbl.second 'ord.output.str := }
        {
          ord.input.str "third" = ord.input.str "3" = or
            { bbl.third 'ord.output.str := }
            {
              ord.input.str "fourth" = ord.input.str "4" = or
                { bbl.fourth 'ord.output.str := }
                {
                  ord.input.str "fifth" = ord.input.str "5" = or
                    { bbl.fifth 'ord.output.str := }
                    {
                      ord.input.str #1 #1 substring$
                      is.a.number
                        { ord.input.str make.ordinal }
                        { ord.input.str }
                      if$
                      'ord.output.str :=
                    }
                   if$
                }
              if$ 
            }
          if$ 
        }
      if$ 
    }
  if$
  ord.output.str
}

% Functions for each type of entry

FUNCTION { format.address } {
  address empty$
    { }
    {
      address
      output
    }
  if$
}

FUNCTION { format.authors } {
  author empty$
    { }
    {
%<*bio>
      author names.separate.comma format.names
%</bio>
%<*!bio>
      author names.separate.semicolon format.names
%</!bio>
      output
      next.punct.space 'next.punct.int :=
    }
  if$   
}

FUNCTION { format.editors } {
  editor empty$
    { }
    {
      editor names.separate.comma format.names
      add.comma
      editor num.names$ #1 >
        { bbl.editors }
        { bbl.editor }
      if$ 
      *
      output
      next.punct.semicolon 'next.punct.int :=
    } 
  if$ 
}

FUNCTION { format.authors.or.editors } {
  author empty$
    { format.editors }
    { format.authors }
  if$ 
  next.punct.space 'next.punct.int :=
}

FUNCTION { format.chapter } {
  chapter empty$
    { }
    {
      bbl.chapter add.space
      chapter
      *
      output
    }
  if$
}

FUNCTION { format.doi } {
  doi empty$
    'skip$
    {
      bbl.doi add.space
      "\doi{" * doi * "}" *
      output
    }
  if$
}

FUNCTION { format.edition } {
  edition empty$
    { }
    {
      edition convert.to.ordinal
      add.space bbl.edition *
      output
    }
  if$ 
  next.punct.semicolon 'next.punct.int :=
}

FUNCTION { format.group.address } {
  duplicate$
  empty$
    { pop$ }
    {
      address empty$
        { }
        {
          add.colon
          address
          *
        }
      if$ 
      output
    }
  if$ 
}

FUNCTION { format.howpublished } {
  howpublished empty$
    { }
    {
      howpublished
      output
    }
  if$
}

FUNCTION { format.journal } {
  journal emph
  output
  next.punct.space 'next.punct.int :=
}

FUNCTION { format.journal.unpub } {
  journal emph
  output
}

FUNCTION { format.note } {
  note empty$
    { }
    {
      note
      output
    }
  if$ 
}

FUNCTION { format.number.series } {
  series empty$
    { }
    {
      series
      number empty$
        { }
        {
          add.space
          number *
        }
      if$ 
      output
      next.punct.semicolon 'next.punct.int :=
    }
  if$ 
}


FUNCTION { format.organization } {
  organization empty$
    { }
    {
      organization paren
      output
      next.punct.period 'next.punct.int :=
    }
  if$ 
}

FUNCTION { format.organization.address } {
  organization format.group.address
}

FUNCTION { format.pages } {
  pages empty$
    { }
    {
      pages multi.page.check
        {
          bbl.pages
          pages hyphen.to.dash
        }
        { bbl.page pages }
      if$
      tie.or.space.connect
      output
    }
  if$
  ctrl.doi.bool
    { format.doi }
    'skip$
  if$
}

FUNCTION { format.pages.article } {
  pages empty$
    { }
    {
      pages hyphen.to.dash
      output
    }
  if$
  ctrl.doi.bool
    { format.doi }
    'skip$
  if$
}

FUNCTION { format.publisher.address } {
  publisher format.group.address
}

FUNCTION { format.school.address } { 
  school
  duplicate$
  empty$
    { pop$ }
    {
      address empty$
        { }
        {
          add.comma
          address
          *
        }
      if$ 
      output
    }
  if$
}

FUNCTION { format.title } {
  title empty$
    { }
    {
      title
      output
      next.punct.period 'next.punct.int :=
    }
  if$
}

FUNCTION { format.title.article } {
  ctrl.article.title.bool
    {
      title empty$
        { }
        {
          title
          output
          next.punct.period 'next.punct.int :=
        }
      if$ 
    }
    { }
  if$
}

FUNCTION { format.title.techreport } {
  title empty$
    { }
    {
      title emph
      output
      next.punct.semicolon 'next.punct.int :=
    }
  if$
}

FUNCTION { format.title.booktitle } {
  title empty$
    { }
    {
      title
      output
      next.punct.period 'next.punct.int :=
    }
  if$ 
  booktitle empty$
    { }
    {
      booktitle
      output
      next.punct.period 'next.punct.int :=
    }
  if$   
}

STRINGS {
  book.title
  chapter.title
}

FUNCTION { format.title.booktitle.book } {
  "" 'chapter.title :=
  booktitle empty$
    {
      "" 'chapter.title :=
      title 'book.title :=
    }
    {
      ctrl.chapter.title.bool
        {
          title empty$
            'skip$
            { title 'chapter.title := }
          if$
        }
        'skip$
      if$
      maintitle empty$
        { booktitle 'book.title := }
        { maintitle add.period booktitle * 'book.title := }
      if$ 
    }
  if$
  chapter.title empty$
    { }
    {
      chapter.title
      output
      next.punct.period 'next.punct.int :=
    }
  if$ 
  book.title emph
  chapter.title empty$
    {
      author empty$
        { }
        {
          editor empty$
            { }
            { bbl.in add.space swap$ * }
          if$  
        }
      if$ 
    }
    { bbl.in add.space swap$ * }
  if$ 
  output
}

FUNCTION { format.type } {
  type empty$
    { }
    {
      pop$
      type
    }
  if$
  output
}

FUNCTION { format.type.number } {
  type empty$
    { }
    {
      type
      number empty$
        { }
        { number tie.or.space.connect }
      if$ 
      output
    }
  if$
}

FUNCTION { format.url } {
  url empty$
    { }
    {
      "\url{" url * "}" *
      output
    }
  if$ 
}

FUNCTION { format.year } {
  year empty$
    { }
    {
      year
      output
      next.punct.semicolon 'next.punct.int :=
    }
  if$ 
}

FUNCTION { format.year.article } {
  year empty$
    { }
    {
%<*bio>
      year paren
      output
      next.punct.space 'next.punct.int :=
%</bio>
%<*!bio>
      year bold
      output
%</!bio>
    }
  if$
}

FUNCTION { format.version } {
  version empty$
    { }
    {
      bbl.version add.space
      version
      *
      output
    } 
  if$ 
}

FUNCTION { format.volume.article } {
  volume emph
  output
}

FUNCTION { format.volume } {
  volume empty$
    { }
    {
      bbl.volume
      volume
      tie.or.space.connect
      output
      next.punct.semicolon 'next.punct.int :=
    }
  if$   
}

% The functions to deal with each entry type

FUNCTION { article } {
  begin.bibitem
  format.authors
%<*bio>
  format.year.article
%</bio>
  format.title.article
  format.journal
%<*!bio>
  format.year.article
%</!bio>
  format.volume.article
  format.pages.article
  format.note
  end.bibitem
}

FUNCTION { book } {
  begin.bibitem
  format.authors.or.editors
  format.title.booktitle.book
  format.edition
  author empty$
    { }
    { format.editors }
  if$
  format.number.series
  format.publisher.address
  format.year
  format.volume
  format.chapter
  format.pages
  format.note
  end.bibitem
}

FUNCTION { inbook } { book }

FUNCTION { booklet } {
  begin.bibitem
  format.authors
  format.title
  format.howpublished
  format.address
  format.year
  format.note
  end.bibitem
}

FUNCTION { collection } { book }

FUNCTION { incollection } { book }

FUNCTION { inpress } {
  begin.bibitem
  format.authors
%<*bio>
  format.year.article
%</bio>
  format.journal.unpub
  doi empty$
    {
      bbl.inpress
      output
    }
    {
%<*!bio>
      format.year.article
%</!bio>
      next.punct.comma 'next.punct.int :=
      format.doi
    }
  if$ 
  format.note
  end.bibitem
}

FUNCTION { inproceedings } {
  begin.bibitem
  format.authors
  format.title.booktitle
  format.address
  format.year
  format.pages
  format.note
  end.bibitem
}

FUNCTION { manual } {
  begin.bibitem
  format.authors
  format.title
  format.version
  format.organization.address
  format.year
  format.note
  end.bibitem
}

FUNCTION { mastersthesis } {
  begin.bibitem
  format.authors
  format.title
  bbl.msc format.type
  format.school.address
  format.year
  format.note
  end.bibitem
}

FUNCTION { misc } {
  begin.bibitem
  format.authors
  format.title
  format.howpublished
  format.year
  format.url
  format.note
  end.bibitem
}

FUNCTION { patent } {
  begin.bibitem
  format.authors
  format.organization
  format.title
  format.type.number
  format.year
  format.note
  end.bibitem
}

FUNCTION { phdthesis } {
  begin.bibitem
  format.authors
  format.title
  bbl.phd format.type
  format.school.address
  format.year
  format.note
  end.bibitem
}

FUNCTION { proceeding } {
  begin.bibitem
  format.title
  format.address
  format.year
  format.pages
  format.note
  end.bibitem
}

FUNCTION { techreport } {
  begin.bibitem
  format.authors.or.editors
  format.title.techreport
  format.type.number
  format.organization.address
  format.year
  format.volume
  format.pages
  format.note
  end.bibitem
}

FUNCTION { unpublished } {
  begin.bibitem
  format.authors
  format.journal.unpub
  doi empty$
    { format.howpublished }
    {
      format.year
      next.punct.comma 'next.punct.int :=
      format.doi
    }
  if$ 
  format.note
  end.bibitem
}

FUNCTION { default.type } { misc }
 
% Macros containing pre-defined short cuts

MACRO { jan } { "Jan." }
MACRO { feb } { "Feb." }
MACRO { mar } { "Mar." }
MACRO { apr } { "Apr." }
MACRO { may } { "May" }
MACRO { jun } { "June" }
MACRO { jul } { "July" }
MACRO { aug } { "Aug." }
MACRO { sep } { "Sept." }
MACRO { oct } { "Oct." }
MACRO { nov } { "Nov." }
MACRO { dec } { "Dec." }

MACRO { acbcct } { "ACS Chem.\ Biol." }
MACRO { achre4 } { "Acc.\ Chem.\ Res." }
MACRO { acncdm } { "ACS Chem.\ Neurosci." }
MACRO { ancac3 } { "ACS Nano" }
MACRO { ancham } { "Anal.\ Chem." }
MACRO { bichaw } { "Biochemistry" }
MACRO { bcches } { "Bioconjugate Chem." }
MACRO { bomaf6 } { "Biomacromolecules" }
MACRO { bipret } { "Biotechnol.\ Prog." }
MACRO { crtoec } { "Chem.\ Res.\ Toxicol." }
MACRO { chreay } { "Chem.\ Rev." }
MACRO { cmatex } { "Chem.\ Mater." }
MACRO { cgdefu } { "Cryst.\ Growth Des." }
MACRO { enfuem } { "Energy Fuels" }
MACRO { esthag } { "Environ.\ Sci.\ Technol." }
MACRO { iechad } { "Ind.\ Eng.\ Chem.\ Res." }
MACRO { inoraj } { "Inorg.\ Chem." }
MACRO { jafcau } { "J.~Agric.\ Food Chem." }
MACRO { jceaax } { "J.~Chem.\ Eng.\ Data" }
MACRO { jceda8 } { "J.~Chem.\ Ed." }
MACRO { jcisd8 } { "J.~Chem.\ Inf.\ Model." }
MACRO { jctcce } { "J.~Chem.\ Theory Comput." }
MACRO { jcchff } { "J. Comb. Chem." }
MACRO { jmcmar } { "J. Med. Chem." }
MACRO { jnprdf } { "J. Nat. Prod." }
MACRO { joceah } { "J.~Org.\ Chem." }
MACRO { jpcafh } { "J.~Phys.\ Chem.~A" }
MACRO { jpcbfk } { "J.~Phys.\ Chem.~B" }
MACRO { jpccck } { "J.~Phys.\ Chem.~C" }
MACRO { jpclcd } { "J.~Phys.\ Chem.\ Lett." }
MACRO { jprobs } { "J.~Proteome Res." }
MACRO { jacsat } { "J.~Am.\ Chem.\ Soc." }
MACRO { langd5 } { "Langmuir" }
MACRO { mamobx } { "Macromolecules" }
MACRO { mpohbp } { "Mol.\ Pharm." }
MACRO { nalefd } { "Nano Lett." }
MACRO { orlef7 } { "Org.\ Lett." }
MACRO { oprdfk } { "Org.\ Proc.\ Res.\ Dev." }
MACRO { orgnd7 } { "Organometallics" }
 
% Construction of bibliography: reading of data, construction of
% labels, output of formatted bibliography

READ

EXECUTE { initialize.control.values }

EXECUTE { initialize.count.entries }

EXECUTE { initialize.name.separator }

EXECUTE { initialize.tracker }

ITERATE { calculate.names }

ITERATE { count.entries }

EXECUTE { begin.bib }

ITERATE { call.type$ }

EXECUTE { end.bib }
%</bst>
%\fi
